% modelo de calidad
% expandirse en ISO 25010

%-utilización de un modelo de calidad
    %-ISO 25010
        %-historia
        %-definiciones
        %-modelo completo
        %-mantenibilidad
        %-otras características que puedan servir para el problema
\chapter{Propuesta}
En este trabajo se propone la generación de un modelo de calidad basado en el 
modelo ISO/IEC 25010, el cual estará enfocado en la mantenibilidad del producto 
de software. Esta característica está directamente relacionada con la 
continuidad de desarrollo del producto.
El modelo ISO/IEC 25010 presenta la mantenibilidad como una de sus principales
características y además entrega un conjunto de sub-características que sirven para
guiar la búsqueda de las métricas apropiadas.
Estas métricas no están contenidas en el modelo y deben ser escogidas por los evaluadores, 
encontrándose muchas de ellas en la literatura.
En este trabajo se presentan las sub-características escogidas así como las 
métricas elegidas para realizar las mediciones sobre un producto de software real, 
con el fin de evaluar su mantenibilidad y así, entregar información acerca como afectaría
su calidad en la continuidad de desarrollo por parte de terceros.
