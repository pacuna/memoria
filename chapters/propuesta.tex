% modelo de calidad
% expandirse en ISO 25010

%-utilización de un modelo de calidad
    %-ISO 25010
    %por que se eligio?
        %-historia
        %-definiciones
        %-modelo completo
        %-mantenibilidad
        %-otras características que puedan servir para el problema
\chapter{Propuesta}
En este trabajo se propone la generación de un modelo de calidad basado en el 
modelo ISO/IEC 25010, el cual estará enfocado en la mantenibilidad del producto 
de software. Esta característica está directamente relacionada con la 
continuidad de desarrollo del producto.
El modelo ISO/IEC 25010 presenta la mantenibilidad como una de sus principales
características y además entrega un conjunto de sub-características que sirven para
guiar la búsqueda de las métricas apropiadas.
Estas métricas no están contenidas en el modelo y deben ser escogidas por los evaluadores, 
encontrándose muchas de ellas en la literatura.
Se presentarán las sub-características escogidas así como las 
métricas para realizar las mediciones sobre un producto de software real, 
con el fin de evaluar su mantenibilidad y así, entregar información acerca como afectaría
su calidad en la continuidad de desarrollo por parte de terceros.

\section{ISO/IEC 25010}

Este estándar define~\cite{25010}:

\begin{enumerate}
\item Un modelo de \textbf{calidad en uso} compuesto de cinco características
relacionadas con el resultado de la interacción cuando un producto es utilizado
en un contexto de uso particular. Este modelo es aplicable al sistema
humano-computador completo, incluyendo los sistemas computaciones en uso
y los productos de software en uso.

\item Un modelo de calidad de producto compuesto de 8 características relacionadas
con las propiedades estáticas de un software y propiedades dinámicas de un sistema
computacional. El modelo es aplicable a los sistemas computacionales y productos
de software.

Como en este caso se desea evaluar un producto de software, específicamente
su mantenibilidad, no se utilizarán características del modelo para calidad
en uso. Sólo se utilizará un conjunto del modelo para calidad de producto.
El interés se enfocará en las propiedades estáticas del sistema, las cuales
a través de su correctitud y adherencia a estándares y buenas prácticas, nos
entregarán información acerca de la mantenibilidad del producto.

A través de otros estándares contenidos en la serie ISO/IEC 25000, se generará
un modelo de calidad y se implementará una evaluación real con la cual se 
pueda certificar la continuidad de desarrollo.

Se trabajará en conjunto con la empresa creadora del producto de software.
Luego de la primera evaluación, se entregará la retroalimentación necesaria
para lograr un nivel más alto de calidad en una evaluación final que permitirá
certificar la mantenibilidad en base al modelo generado.
\end{enumerate}
