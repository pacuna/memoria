% modelo de calidad
% expandirse en ISO 25010

%-utilización de un modelo de calidad
    %-ISO 25010
    %por que se eligio?
        %-historia
        %-definiciones
        %-modelo completo
        %-mantenibilidad
        %-otras características que puedan servir para el problema
\chapter{Propuesta}

La serie de estándares ISO/IEC 25000 se denomina SQuaRE (Software product
Quality Requirements and Evaluation) y se compone de la siguientes divisiones~\cite{25000}:

\begin{itemize}
    \item ISO/IEC 2500n - División de Gestión de Calidad
    \item ISO/IEC 2501n - División de Modelo de Calidad
    \item ISO/IEC 2502n - División de Medición de Calidad
    \item ISO/IEC 2503n - División de Requerimientos de Calidad
    \item ISO/IEC 2504n - División de Evaluación de Calidad
\end{itemize}

Cada una de estas divisiones entrega estándares y guías para realizar
el análisis de calidad correspondiente. 

En SQuaRE se entregan:
\begin{itemize}
    \item Términos y definiciones
    \item Modelos de referencia
    \item Guía general
    \item Guías individuales para cada división
    \item Estándares para propósitos como especificación de requerimientos, 
    planeación y gestión, medición y evaluación.
\end{itemize}

SQuaRE reemplaza a las series ISO/IEC 9126 y 14598.
Para esta memoria se utilizaron la división de Modelo de Calidad, de la cual
se obtuvo un conjunto de características enfocadas en la mantenibilidad del 
producto, y la división de Evaluación de Calidad, de la cual se obtuvieron
guías para aplicar el modelo obtenido y hacer una evaluación de calidad
sobre un producto de software real.

Se propone la generación de un modelo de calidad basado en el 
modelo ISO/IEC 25010, el cual estará enfocado en la mantenibilidad del producto 
de software. Esta característica está directamente relacionada con la 
continuidad de desarrollo del producto.
El modelo ISO/IEC 25010 presenta además la mantenibilidad como una de sus principales
características y además entrega un conjunto de sub-características que sirven para
guiar la búsqueda de las métricas apropiadas.
Estas métricas no están contenidas en el modelo y deben ser escogidas por los evaluadores, 
encontrándose muchas de ellas en la literatura.
Se presentarán las sub-características escogidas así como las 
métricas para realizar las mediciones sobre un producto de software real, 
con el fin de evaluar su mantenibilidad y así, entregar información acerca como afectaría
su calidad en la continuidad de desarrollo por parte de terceros.

%sobre ISO 25010
\section{ISO/IEC 25010}

Este estándar define~\cite{25010}:

\begin{enumerate}
\item Un modelo de \textbf{calidad en uso} compuesto de cinco características
relacionadas con el resultado de la interacción cuando un producto es utilizado
en un contexto de uso particular. Este modelo es aplicable al sistema
humano-computador completo, incluyendo los sistemas computaciones en uso
y los productos de software en uso.

\item Un modelo de \textbf{calidad de producto} compuesto de 8 características relacionadas
con las propiedades estáticas de un software y propiedades dinámicas de un sistema
computacional. El modelo es aplicable a los sistemas computacionales y productos
de software.

Como en este caso se desea evaluar un producto de software, específicamente
su mantenibilidad, no se utilizarán características del modelo para calidad
en uso. Sólo se utilizará un conjunto del modelo para calidad de producto.
El interés se enfocará en las propiedades estáticas del sistema, las cuales
a través de su correctitud y adherencia a estándares y buenas prácticas, nos
entregarán información acerca de la mantenibilidad del producto.

A través de otros estándares contenidos en la serie ISO/IEC 25000, se generará
un modelo de calidad y se implementará una evaluación real a través de la cual se 
pueda certificar la continuidad de desarrollo.

Se trabajará en conjunto con la empresa creadora del producto de software.
Luego de la primera evaluación, se entregará la retroalimentación necesaria
para lograr un nivel más alto de calidad en una evaluación final que permitirá
certificar la mantenibilidad en base al modelo generado.
\end{enumerate}

\subsection{Modelo de Calidad de Producto}
Este modelo se compone de las siguientes características y sub-características:

\begin{itemize}
\item Adecuación Funcional
    \begin{itemize}
        \item Completitud funcional
        \item Correctitud funcional
        \item Adecuidad funcional
    \end{itemize}
\item Eficiencia de desempeño
    \begin{itemize}
        \item Comportamiento en el tiempo
        \item Utilización de recursos
        \item Capacidad
    \end{itemize}
\item Compatibilidad
    \begin{itemize}
        \item Co-existencia
        \item Interoperabilidad
    \end{itemize}
\item Usabilidad
    \begin{itemize}
        \item Reconocimiento de su adecuación
        \item Capacidad de ser aprendido
        \item Protección de error para el usuario
        \item Estética de interfaz de usuario
        \item Accesibilidad
    \end{itemize}
\item Confiabilidad
    \begin{itemize}
        \item Madurez
        \item Disponibilidad
        \item Tolerancia a fallos
        \item Capacidad de recuperación
    \end{itemize}
\item Seguridad
    \begin{itemize}
        \item Confidencialidad
        \item Integridad
        \item No-repudio
        \item Responsabilidad
        \item Autenticidad
    \end{itemize}
\item Mantenibilidad
    \begin{itemize}
        \item Modularidad
        \item Reusabilidad
        \item Analizabilidad
        \item Modificabilidad
        \item Capacidad de pruebas
    \end{itemize}
\item Portabilidad
    \begin{itemize}
        \item Adaptabilidad
        \item Instalabilidad
        \item Capacidad de ser reemplazado
    \end{itemize}
\end{itemize}

Este modelo es útil para especificar requerimientos, establecer mediciones y 
realizar evaluaciones de calidad. Las características definidas pueden ser utilizadas
como una lista de verificación para asegurar un tratamiento exhaustivo de
los requerimientos de calidad.

En la práctica es muy complicado medir todas las subcaracterísticas para un sistema
o producto de software de gran tamaño. De esta manera, la importancia de las 
características dependerán de los objetivos y metas del proyecto. El modelo
debe ser adaptado antes de su uso como parte de la descomposición de requerimientos
para identificar aquellas características y sub-características que son más
importantes.

Este trabajo se enfocará solamente en la continuidad de desarrollo del producto
de software, por lo que la mayor cantidad de esfuerzo en las mediciones, se
realizará en la característica Mantenibilidad. Esta característica tiene
una fuerte influencia en la calidad de uso para las personas que realizarán
tareas de mantención en el software. Estas tareas no sólo serán de mantenimiento,
sino también extender el software, arreglar defectos, realizar inspecciones
de código, etc.

En la siguiente sección se presentará la implementación de la propuesta
utilizando el modelo.
