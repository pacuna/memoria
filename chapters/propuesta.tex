% modelo de calidad
% expandirse en ISO 25010

%-utilización de un modelo de calidad
    %-ISO 25010
    %por que se eligio?
        %-historia
        %-definiciones
        %-modelo completo
        %-mantenibilidad
        %-otras características que puedan servir para el problema
\chapter{Propuesta}

La serie de estándares ISO/IEC 25000 se denomina SQuaRE (Software product
Quality Requirements and Evaluation) y se compone de la siguientes divisiones~\cite{25000}:

\begin{itemize}
    \item ISO/IEC 2500n - División de Gestión de Calidad
    \item ISO/IEC 2501n - División de Modelo de Calidad
    \item ISO/IEC 2502n - División de Medición de Calidad
    \item ISO/IEC 2503n - División de Requerimientos de Calidad
    \item ISO/IEC 2504n - División de Evaluación de Calidad
\end{itemize}

Cada una de estas divisiones entrega estándares y guías para realizar
el análisis de calidad correspondiente. 

En SQuaRE se entregan:
\begin{itemize}
    \item Términos y definiciones
    \item Modelos de referencia
    \item Guía general
    \item Guías individuales para cada división
    \item Estándares para propósitos como especificación de requerimientos, 
    planeación y gestión, medición y evaluación.
\end{itemize}

SQuaRE reemplaza a las series ISO/IEC 9126 y 14598.
Para esta memoria se utilizaron la división de Modelo de Calidad, de la cual
se obtuvo un conjunto de características enfocadas en la mantenibilidad del 
producto, y la división de Evaluación de Calidad, de la cual se obtuvieron
guías para aplicar el modelo obtenido y hacer una evaluación de calidad
sobre un producto de software real.

Se propone la generación de un modelo de calidad basado en el 
modelo ISO/IEC 25010, el cual estará enfocado en la mantenibilidad del producto 
de software. Esta característica está directamente relacionada con la 
continuidad de desarrollo del producto.
El modelo ISO/IEC 25010 presenta además la mantenibilidad como una de sus principales
características y además entrega un conjunto de sub-características que sirven para
guiar la búsqueda de las métricas apropiadas.
Estas métricas no están contenidas en el modelo y deben ser escogidas por los evaluadores, 
encontrándose muchas de ellas en la literatura.
Se presentarán las sub-características escogidas así como las 
métricas para realizar las mediciones sobre un producto de software real, 
con el fin de evaluar su mantenibilidad y así, entregar información acerca como afectaría
su calidad en la continuidad de desarrollo por parte de terceros.

%sobre ISO 25010
\section{ISO/IEC 25010}

Este estándar define~\cite{25010}:

\begin{enumerate}
\item Un modelo de \textbf{calidad en uso} compuesto de cinco características
relacionadas con el resultado de la interacción cuando un producto es utilizado
en un contexto de uso particular. Este modelo es aplicable al sistema
humano-computador completo, incluyendo los sistemas computaciones en uso
y los productos de software en uso.

\item Un modelo de \textbf{calidad de producto} compuesto de 8 características relacionadas
con las propiedades estáticas de un software y propiedades dinámicas de un sistema
computacional. El modelo es aplicable a los sistemas computacionales y productos
de software.

Como en este caso se desea evaluar un producto de software, específicamente
su mantenibilidad, no se utilizarán características del modelo para calidad
en uso. Sólo se utilizará un conjunto del modelo para calidad de producto.
El interés se enfocará en las propiedades estáticas del sistema, las cuales
a través de su correctitud y adherencia a estándares y buenas prácticas, nos
entregarán información acerca de la mantenibilidad del producto.

A través de otros estándares contenidos en la serie ISO/IEC 25000, se generará
un modelo de calidad y se implementará una evaluación real a través de la cual se 
pueda certificar la continuidad de desarrollo.

Se trabajará en conjunto con la empresa creadora del producto de software.
Luego de la primera evaluación, se entregará la retroalimentación necesaria
para lograr un nivel más alto de calidad en una evaluación final que permitirá
certificar la mantenibilidad en base al modelo generado.
\end{enumerate}

\subsection{Modelo de Calidad de Producto}
Este modelo se compone de las siguientes características y sub-características:

\begin{itemize}
\item Adecuación Funcional
    \begin{itemize}
        \item Completitud funcional
        \item Correctitud funcional
        \item Adecuidad funcional
    \end{itemize}
\item Eficiencia de desempeño
    \begin{itemize}
        \item Comportamiento en el tiempo
        \item Utilización de recursos
        \item Capacidad
    \end{itemize}
\item Compatibilidad
    \begin{itemize}
        \item Co-existencia
        \item Interoperabilidad
    \end{itemize}
\item Usabilidad
    \begin{itemize}
        \item Reconocimiento de su adecuación
        \item Capacidad de ser aprendido
        \item Protección de error para el usuario
        \item Estética de interfaz de usuario
        \item Accesibilidad
    \end{itemize}
\item Confiabilidad
    \begin{itemize}
        \item Madurez
        \item Disponibilidad
        \item Tolerancia a fallos
        \item Capacidad de recuperación
    \end{itemize}
\item Seguridad
    \begin{itemize}
        \item Confidencialidad
        \item Integridad
        \item No-repudio
        \item Responsabilidad
        \item Autenticidad
    \end{itemize}
\item Mantenibilidad
    \begin{itemize}
        \item Modularidad
        \item Reusabilidad
        \item Analizabilidad
        \item Modificabilidad
        \item Capacidad de pruebas
    \end{itemize}
\item Portabilidad
    \begin{itemize}
        \item Adaptabilidad
        \item Instalabilidad
        \item Capacidad de ser reemplazado
    \end{itemize}
\end{itemize}

Este modelo es útil para especificar requerimientos, establecer mediciones y 
realizar evaluaciones de calidad. Las características definidas pueden ser utilizadas
como una lista de verificación para asegurar un tratamiento exhaustivo de
los requerimientos de calidad.

En la práctica es muy complicado medir todas las subcaracterísticas para un sistema
o producto de software de gran tamaño. De esta manera, la importancia de las 
características dependerán de los objetivos y metas del proyecto. El modelo
debe ser adaptado antes de su uso como parte de la descomposición de requerimientos
para identificar aquellas características y sub-características que son más
importantes.

Este trabajo se enfocará solamente en la continuidad de desarrollo del producto
de software, por lo que la mayor cantidad de esfuerzo en las mediciones, se
realizará en la característica Mantenibilidad. Esta característica tiene
una fuerte influencia en la calidad de uso para las personas que realizarán
tareas de mantención en el software. Estas tareas no sólo serán de mantenimiento,
sino también extender el software, arreglar defectos, realizar inspecciones
de código, etc.

En la siguiente sección se presentará la implementación de la propuesta
utilizando el modelo.

\section{Definición del modelo}

Para evaluar un producto de software utilizando el modelo ISO/IEC 25010, se debe escoger un sub-conjunto
de características de las principales que lo componen. Estas deben ser escogidas de acuerdo
al motivo de la evaluación y deben ser acordes al contexto del producto.

Como se mencionó previamente, la continuidad de desarrollo de un producto de software está directamente
ligada con la mantenibilidad del producto, puesto que la calidad de su código fuente va a incidir
en el trabajo posterior de los desarrolladores, más aún si estos no tienen conocimiento previo del sistema.

Para realizar esta elección, la ISO/IEC provee guías con una serie de consejos para definir correctamente
un plan de calidad, partiendo por la elección de características y finalizando con el análisis de la evaluación
propiamente tal. Estas guías se pueden encontrar en la división ISO/IEC 25040~\cite{25040}, la cual corresponde a la división
de evaluación de calidad. Esta provee requerimientos, recomendaciones y guías para
la evaluación de un producto de software ya sean realizadas por evaluadores independientes, adquisidores o
desarrolladores. También se presenta un apoyo para documentar una medición como un módulo de evaluación.

Los principales hitos dentro de este proceso se pueden resumir en:
\begin{enumerate}
    \item Establecer los requerimientos de la evaluación
    \item Especificar la evaluación
    \item Diseñar la evaluación
    \item Ejecutar la evaluación
    \item Concluir la evaluación
\end{enumerate}

La ejecución de estas tareas en un producto de software real se describe a continuación.

\section{Requerimientos de la evaluación}

\subsection{Propósito de la evaluación}

Se desea evaluar el producto de software SAE de la empresa MOSAQ con el fin de
elaborar un plan de calidad que permita cerrar brechas en torno a temas de 
mantenibilidad de su producto. A través de esta evaluación se acreditará que la 
empresa utilizó un conjunto de buenas prácticas para construir el software y que 
éste puede ser mantenido y modificado por terceros en el caso que fuese necesario.

\subsection{Requerimientos de calidad del producto de software}

En conjunto con MOSAQ, se analizaron las características y subcaracterísticas que ofrece
el modelo ISO 25010 con el fin de encontrar las que mejor se adecuen a los requerimientos
de calidad. El principal requerimiento es mantenibilidad y además se optó por evaluar
algunos aspectos de la portabilidad del producto. Para estas dos características, se
escogieron subcaracterísticas del modelo ISO 25010 y ciertas métricas para implementar
la evaluación.

\subsection{Partes del producto sometidas a evaluación}

Se sometieron a evaluación dos módulos del producto de software. Estos son SAE Framework
Servidor y SAE Framework Cliente. La evaluación difiere en algunos aspectos para cada módulo
puesto que tienen diferencias en la arquitectura de la implementación.

\subsection{Rigor de la evaluación}

Se decidió que la mayor cantidad de esfuerzo y rigor debe estar enfocado en estudiar y
analizar la mantenibilidad del producto. 

Se estudiaron a fondo métricas relacionadas con la mantenibilidad del producto y se
utilizaron criterios que permiten asegurar prácticas profesionales en la construcción
del producto de software.

También se realizaron descripciones cualitativas acerca de la portabilidad del producto.
Esta característica junto con la manteniblidad, es importante para los clientes de MOSAQ
ya que ambas influyen en los procesos que se deben llevar a cabo en el caso de que se tuviese
que trabajar con terceros en un futuro.

\section{Especificación de la evaluación para SAEFramework Servidor}
%Especificacion para SAE Servidor

A continuación se presentan las métricas definidas para realizar la evaluación. Estas métricas 
han sido seleccionadas tomando en cuenta las principales recomendaciones que la ISO entrega en 
la serie 25000, las cuales sirven para implementar un modelo y plan de evaluación de calidad 
utilizando el modelo presentado en la división 25010.


\subsection{Selección de métricas para mantenibilidad}
Las subcaracterísticas elegidas para mantenibilidad son \textbf{modularidad}, \textbf{reusabilidad}, 
\textbf{modificabilidad} y \textbf{capacidad de pruebas}.

A continuación se presentan las métricas para evaluar estas subcaracterísticas.

\subsubsection{Cohesión Relacional (Modularidad)}
Es el número promedio de relaciones internas por tipo. Se mide utilizando:
\begin{equation*}
H=\frac{R+1}{N}
\end{equation*}
Donde $R$ es el número de relaciones internas entre tipos y el paquete, $N$ el número de tipos en el paquete.

Las clases dentro de un \textit{assembly}\footnote{biblioteca de código compilado} deben estar fuertemente 
relacionadas, de esta manera la cohesión tendrá tener un valor alto. Por otro lado, valores demasiado altos 
podrían indicar sobre-acoplamiento. Un buen rango es $1.5\leq H\leq 4.0$.

%FIXME: sugerencia: el debería cambiarlo a debe. definir en nota al pie un assembly (arreglado)

\subsubsection{LCOM (Falta de cohesión en métodos) (Modularidad)}
El principio de responsabilidad única consiste en que una clase no debe tener más de una razón para cambiar. 
Una clase con esta característica es cohesiva.

\begin{equation*}
LCOM = 1 - \frac{\sum_{f\in F}\left|M_f\right|}{\left|M\right|\times\left|F\right|}
\end{equation*}

Donde $M$ son los métodos estáticos e instancias en la clase, $F$ campos instanciados en la clase y $M_f$ 
los métodos que acceden el campo $f_i$.

En una clase que es completamente cohesionada, cada método debe acceder a cada campo instanciado:
\begin{equation*}
\sum_f \left|M_f\right| = \left|M\right|\times\left|F\right|
\end{equation*}
de manera que el $LCOM=0$.

Un valor alto de $LCOM$ generalmente quiere decir que una clase tiene una baja cohesión. Tipos en los cuales
$LCOM\ge 0.8$ y $\left|F\right|\ge 10$ y $\left|M\right|\ge 10$ podrían ser problemáticos. Sin embargo, es 
muy difícil evitar estos casos con poca cohesión.

\subsubsection{LCOM HS (Falta de cohesión de métodos Henderson-Sellers) (Modularidad)}
Esta métrica es similar a la anterior, pero toma su valor en un rango $\left[ 0-2\right]$. Un valor LCOM 
HS mayor a $1$ debería ser considerado peligroso.

\begin{equation*}
LCOM HS = M - \frac{\sum_{f\in F}\left|M_f\right|}{F}\times (M-1)
\end{equation*}
Tipos en los cuales $LCOM HS\ge 1.0$ y $\left|F\right|\ge 10$ y $\left|M\right|\ge 10$ deberían ser evitados. 
Esta restricción es más fuerte (por lo tanto más fácil de satisfacer) que la descrita para $LCOM$.

\subsubsection{Acoplamiento eferente (Modularidad)}
Número de tipos en el paquete correspondiente, que dependen de tipos que están fuera del paquete.

Un valor muy alto de esta métrica podría implicar problemas de diseño. Tipos que tengan este valor muy 
alto están entrelazados con muchas otras implementaciones. Mientras más alto sea el valor, mayor es el número  
de responsabilidades que el tipo tiene.

\subsubsection{Acoplamiento aferente (Modularidad)}
Número de tipos fuera del paquete, que dependen de tipos que están en el paquete en evaluación. 

Un valor alto de esta métrica no es necesariamente peligroso, sin embargo es interesante saber que partes 
del código son altamente utilizadas.

Esta métrica es útil especialmente cuando es igual a $0$, lo cual podría indicar un elemento de código sin 
uso. Estos casos deben ser manejados con cuidado para puntos de entrada, constructores de clases o 
finalizadores ya que estos métodos siempre tendrán un valor $0$ para acoplamiento aferente y no 
corresponden a código sin uso.

\subsubsection{Instabilidad (Modificabilidad)}

Es la razón entre el acoplamiento eferente y el acoplamiento total. Esta métrica indica la resiliencia 
al cambio del paquete.

\begin{equation*}
I = C_e / (C_e + C_a)
\end{equation*}

Donde $C_e$ es el acoplamiento eferente y $C_a$ el acoplamiento aferente.

Un valor de $I=0$ indica un paquete completamente estable, fácil de modificar. Un valor de $I=1$ indica 
un paquete completamente inestable.

\subsubsection{Complejidad Ciclomática (Modificabilidad)}
Número de decisiones que pueden ser tomadas en un procedimiento.
Procedimientos con un valor mayor a 15 son difíciles de entender, mientras que con un valor mayor a 30 
son extremadamente complejos y deberían ser divididos en métodos más pequeños (a menos que sea código 
auto-generado).

\subsubsection{índice de mantenibilidad (Modificabilidad)}

Corresponde a un índice entre 0 y 100 que representa la facilitad relativa
de mantener el código. Un valor más alto indica una mejor mantenibilidad. Un valor entre 20 y 100 
indica que el código tiene una buena mantenibilidad. Un valor entre 10 y 19 indica que el código es 
moderadamente mantenible y un código entre 0 y 9 indica una baja mantenibilidad.

\subsubsection{Código duplicado (Reusabilidad)}

Se utiliza alguna heurística para detectar código potencialmente duplicado.
El hecho de encontrar un porcentaje alto de duplicación, podría indicar que no se está haciendo un reuso 
adecuado en el software.

\subsection{Selección de métricas para portabilidad}

\subsubsection{Prácticas de instalación (Instalabilidad)}

Descripciones cualitativas acerca de cómo se implementa una instalación estándar para el producto. 
Se estudian estos procesos y se recomiendan mejoras para alinear las prácticas a estándares profesionales.
%Especificacion para SAE Cliente

\section{Especificación de la evaluación para SAE Framework Cliente}
\subsection{Selección de métricas para mantenibilidad}
\subsubsection{Complejidad Ciclomática (Modificabilidad)}
Número de decisiones que pueden ser tomadas en un procedimiento.
Procedimientos con un valor mayor a 15 son difíciles de entender, mientras que con un valor mayor a 
30 son extremadamente complejos y deberían ser divididos en métodos más pequeños (a menos que sea 
código auto-generado).

\subsubsection{Profundidad (Modificabilidad, Analizabilidad)}
Estudia el nivel de anidamiento que puede existir entre funciones o expresiones dentro del código. 
Se debe definir un límite crítico de profundidad y verificar que no se esté sobrepasando.

\subsubsection{Número de parámetros (Analizabilidad)}
Estudia el número de parámetros en una función.
Al reducir este valor, se puede mejorar la analizabilidad y modularidad del código de manera sustancial.
Al igual que la métrica anterior, se debe definir un valor límite para estudiar el código y verificar que
no se esté sobrepasando.

\subsubsection{Código duplicado (Reusabilidad)}

Se utiliza alguna heurística para detectar código potencialmente duplicado.
El hecho de encontrar un porcentaje alto de duplicación, podría indicar que no se está haciendo un reuso 
adecuado en el software.

%----FIN METRICAS PARA SAE CLIENTE---------------------------------------------------------%
\section{Diseño de la evaluación}
Para llevar a cabo las mediciones, MOSAQ hizo entrega de las fuentes principales de su software, 
los cuales bajo un acuerdo de confidencialidad, fueron analizados por parte del equipo Toeska.

\subsection{Para SAEFramework Servidor}
Para este módulo se utilizaron las siguientes herramientas para realizar las mediciones.

\begin{itemize}
\item NDepend v5.0.0.8085\footnote{http://www.ndepend.com}
\item Visual Studio 2013 Code Metrics\footnote{http://msdn.microsoft.com/en-us/library/bb385914.aspx}
\item Visual Studio 2012 Code Clone Analysis\footnote{http://msdn.microsoft.com/en-us/library/hh205279.aspx}
\end{itemize}

\subsection{Para SAEFramework Cliente}

\begin{itemize}
\item SonarQube\footnote{http://www.sonarqube.org/}
\item JsHint v2.1.11\footnote{http://www.jshint.com/}
\item WebStorm v7 Inspection tools\footnote{http://www.jetbrains.com/webstorm/}


La configuración elegida para JsHint es la siguiente:
\begin{verbatim}
{
    "globals": {
        "console": false,
        "jQuery": false,
        "_": false
    },
    "maxparams": 5,
    "maxdepth": 5,
    "maxstatements": 25,
    "maxcomplexity": 10,
    "es5": true,
    "browser": true,
    "boss": false,
    "curly": false,
    "debug": false,
    "devel": false,
    "eqeqeq": true,
    "evil": true,
    "forin": false,
    "immed": true,
    "laxbreak": false,
    "newcap": true,
    "noarg": true,
    "noempty": false,
    "nonew": false,
    "nomen": false,
    "onevar": true,
    "plusplus": false,
    "regexp": false,
    "undef": true,
    "sub": true,
    "strict": false,
    "white": true,
    "unused": true
}
\end{verbatim}

En esta configuración podemos ver los parámetros \textit{maxparams}, \textit{maxdepth} 
y \textit{maxcomplexity} los cuales nos permiten definir los límites para el número de parámetros, 
profundidad y complejidad ciclomática respectivamente. Estos parámetros fueron seleccionados de manera 
informada de acuerdo a buenas prácticas investigadas previamente y acordes a un software con las 
características de SAE Framework.
\end{itemize}

% PLAN DE ACTIVIDADES DE EVALUACION %
\subsection{Plan de actividades de evaluación}

A través de reuniones con MOSAQ, se acordaron los siguientes puntos dentro del plan de trabajo en lo 
que respecta a planificación y evaluación de calidad.
\begin{itemize}
\item El equipo Toeska realizará las mediciones pertinentes sobre el código fuente y sobre algunas 
prácticas de desarrollo del equipo de MOSAQ. Con estas mediciones se evaluarán los requerimientos de 
calidad establecidos en el modelo que se definió.

\item El equipo Toeska entregará los resultados de la primera evaluación, los puntos donde se encontraron 
brechas que deben ser consideradas, además guías para realizar los ajustes necesarios para que el producto 
obtenga mejores resultados durante la próxima evaluación.

\item Finalmente se realizará otra evaluación utilizando los mismo criterios utilizados 
para la primera evaluación, en la cual se espera haber cerrado las brechas encontradas anteriormente 
y así certificar que el producto cumple con las características escogidas de la norma ISO 25010.
\end{itemize}

En el capítulo~\ref{chap:validation} se presentan los resultados obtenidos al aplicar el modelo
generado en el producto de software.
