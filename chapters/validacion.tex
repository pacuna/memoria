\chapter{Validación}
\label{chap:validacion}
\section{Retroalimentación}

Para obtener retroalimentación de la empresa que solicitó la evaluación, se elaboró una
pequeña encuesta y se entregó al arquitecto de software.
A continuación se presentan los resultados.

El arquitecto de software entregó las siguientes respuestas:
\begin{enumerate}
    \item ¿Cree que este tipo de evaluaciones entrega resultados fiables acerca de la mantenibilidad de un producto de software? ¿Por qué?

En la medida que el proceso de evaluación considere parámetros reales de evaluación y
se haga un estudio minucioso, los resultados tenderán a ser fiables,
esto utiliza técnicas y procedimientos probados en la industria que han demostrado confiabilidad.

\item ¿Cree que los resultados de esta evaluación de calidad reflejan las características reales de su producto de software?

Claro, la prueba de esto es la curva de aprendizaje de nuevos desarrolladores,
los que hasta ahora han podido adoptar el producto de manera muy fácil.

\item ¿De que forma cree que los resultados pueden ayudar a perfeccionar el desarrollo dentro de su empresa?

Los nuevos productos de software que estamos realizando están empleando (o emplearán)
esta misma estrategia de desarrollo ya que ha demostrado muchos beneficios
respecto de los desarrollos tradicionales realizados en el pasado.
\end{enumerate}
