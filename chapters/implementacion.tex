%como se armó el modelo
\chapter{Implementación}
Para evaluar un producto de software utilizando el modelo ISO/IEC 25010, se debe escoger un sub-conjunto
de características de las principales que lo componen. Estas deben ser escogidas de acuerdo
al motivo de la evaluación y deben ser acordes al contexto del producto.

Como se mencionó previamente, la continuidad de desarrollo de un producto de software está directamente
ligada con la mantenibilidad del producto, puesto que la calidad de su código fuente va a incidir
en el trabajo posterior de los desarrolladores, más aún si estos no tienen conocimiento previo del sistema.

Para realizar esta elección, la ISO/IEC provee guías con una serie de consejos para definir correctamente
un plan de calidad, partiendo por la elección de características y finalizando con el análisis de la evaluación
propiamente tal. Estas guías se pueden encontrar en la división ISO/IEC 25040~\cite{25040}, la cual corresponde a la división
de evaluación de calidad. Esta provee requerimientos, recomendaciones y guías para
la evaluación de un producto de software ya sean realizadas por evaluadores independientes, adquisidores o
desarrolladores. También se presenta un apoyo para documentar una medición como un módulo de evaluación.

Los principales hitos dentro de este proceso se pueden resumir en:
\begin{enumerate}
    \item Establecer los requerimientos de la evaluación
    \item Especificar la evaluación
    \item Diseñar la evaluación
    \item Ejecutar la evaluación
    \item Concluir la evaluación
\end{enumerate}

La ejecución de estas tareas en un producto de software real se describe a continuación.

\section{Requerimientos de la evaluación}

\subsection{Propósito de la evaluación}

Se desea evaluar el producto de software SAE de la empresa MOSAQ con el fin de
elaborar un plan de calidad que permita cerrar brechas en torno a temas de 
mantenibilidad de su producto. A través de esta evaluación se acreditará que la 
empresa utilizó un conjunto de buenas prácticas para construir el software y que 
éste puede ser mantenido y modificado por terceros en el caso que fuese necesario.

\subsection{Requerimientos de calidad del producto de software}

En conjunto con MOSAQ, se analizaron las características y subcaracterísticas que ofrece
el modelo ISO 25010 con el fin de encontrar las que mejor se adecuen a los requerimientos
de calidad. El principal requerimiento es mantenibilidad y además se optó por evaluar
algunos aspectos de la portabilidad del producto. Para estas dos características, se
escogieron subcaracterísticas del modelo ISO 25010 y ciertas métricas para implementar
la evaluación.

\subsection{Partes del producto sometidas a evaluación}

Se sometieron a evaluación dos módulos del producto de software. Estos son SAE Framework
Servidor y SAE Framework Cliente. La evaluación difiere en algunos aspectos para cada módulo
puesto que tienen diferencias en la arquitectura de la implementación.

\subsection{Rigor de la evaluación}

Se decidió que la mayor cantidad de esfuerzo y rigor debe estar enfocado en estudiar y
analizar la mantenibilidad del producto. 

Se estudiaron a fondo métricas relacionadas con la mantenibilidad del producto y se
utilizaron criterios que permiten asegurar prácticas profesionales en la construcción
del producto de software.

También se realizaron descripciones cualitativas acerca de la portabilidad del producto.
Esta característica junto con la manteniblidad, es importante para los clientes de MOSAQ
ya que ambas influyen en los procesos que se deben llevar a cabo en el caso de que se tuviese
que trabajar con terceros en un futuro.

\section{Especificación de la evaluación para SAEFramework Servidor}
%Especificacion para SAE Servidor

A continuación se presentan las métricas definidas para realizar la evaluación. Estas métricas 
han sido seleccionadas tomando en cuenta las principales recomendaciones que la ISO entrega en 
la serie 25000, las cuales sirven para implementar un modelo y plan de evaluación de calidad 
utilizando el modelo presentado en la división 25010.


\subsection{Selección de métricas para mantenibilidad}
Las subcaracterísticas elegidas para mantenibilidad son \textbf{modularidad}, \textbf{reusabilidad}, 
\textbf{modificabilidad} y \textbf{capacidad de pruebas}.

A continuación se presentan las métricas para evaluar estas subcaracterísticas.

\subsubsection{Cohesión Relacional (Modularidad)}
Es el número promedio de relaciones internas por tipo. Se mide utilizando:
\begin{equation*}
H=\frac{R+1}{N}
\end{equation*}
Donde $R$ es el número de relaciones internas entre tipos y el paquete, $N$ el número de tipos en el paquete.

Las clases dentro de un \textit{assembly}\footnote{biblioteca de código compilado} deben estar fuertemente 
relacionadas, de esta manera la cohesión tendrá tener un valor alto. Por otro lado, valores demasiado altos 
podrían indicar sobre-acoplamiento. Un buen rango es $1.5\leq H\leq 4.0$.

%FIXME: sugerencia: el debería cambiarlo a debe. definir en nota al pie un assembly (arreglado)

\subsubsection{LCOM (Falta de cohesión en métodos) (Modularidad)}
El principio de responsabilidad única consiste en que una clase no debe tener más de una razón para cambiar. 
Una clase con esta característica es cohesiva.

\begin{equation*}
LCOM = 1 - \frac{\sum_{f\in F}\left|M_f\right|}{\left|M\right|\times\left|F\right|}
\end{equation*}

Donde $M$ son los métodos estáticos e instancias en la clase, $F$ campos instanciados en la clase y $M_f$ 
los métodos que acceden el campo $f_i$.

En una clase que es completamente cohesionada, cada método debe acceder a cada campo instanciado:
\begin{equation*}
\sum_f \left|M_f\right| = \left|M\right|\times\left|F\right|
\end{equation*}
de manera que el $LCOM=0$.

Un valor alto de $LCOM$ generalmente quiere decir que una clase tiene una baja cohesión. Tipos en los cuales
$LCOM\ge 0.8$ y $\left|F\right|\ge 10$ y $\left|M\right|\ge 10$ podrían ser problemáticos. Sin embargo, es 
muy difícil evitar estos casos con poca cohesión.

\subsubsection{LCOM HS (Falta de cohesión de métodos Henderson-Sellers) (Modularidad)}
Esta métrica es similar a la anterior, pero toma su valor en un rango $\left[ 0-2\right]$. Un valor LCOM 
HS mayor a $1$ debería ser considerado peligroso.

\begin{equation*}
LCOM HS = M - \frac{\sum_{f\in F}\left|M_f\right|}{F}\times (M-1)
\end{equation*}
Tipos en los cuales $LCOM HS\ge 1.0$ y $\left|F\right|\ge 10$ y $\left|M\right|\ge 10$ deberían ser evitados. 
Esta restricción es más fuerte (por lo tanto más fácil de satisfacer) que la descrita para $LCOM$.

\subsubsection{Acoplamiento eferente (Modularidad)}
Número de tipos en el paquete correspondiente, que dependen de tipos que están fuera del paquete.

Un valor muy alto de esta métrica podría implicar problemas de diseño. Tipos que tengan este valor muy 
alto están entrelazados con muchas otras implementaciones. Mientras más alto sea el valor, mayor es el número  
de responsabilidades que el tipo tiene.

\subsubsection{Acoplamiento aferente (Modularidad)}
Número de tipos fuera del paquete, que dependen de tipos que están en el paquete en evaluación. 

Un valor alto de esta métrica no es necesariamente peligroso, sin embargo es interesante saber que partes 
del código son altamente utilizadas.

Esta métrica es útil especialmente cuando es igual a $0$, lo cual podría indicar un elemento de código sin 
uso. Estos casos deben ser manejados con cuidado para puntos de entrada, constructores de clases o 
finalizadores ya que estos métodos siempre tendrán un valor $0$ para acoplamiento aferente y no 
corresponden a código sin uso.

\subsubsection{Instabilidad (Modificabilidad)}

Es la razón entre el acoplamiento eferente y el acoplamiento total. Esta métrica indica la resiliencia 
al cambio del paquete.

\begin{equation*}
I = C_e / (C_e + C_a)
\end{equation*}

Donde $C_e$ es el acoplamiento eferente y $C_a$ el acoplamiento aferente.

Un valor de $I=0$ indica un paquete completamente estable, fácil de modificar. Un valor de $I=1$ indica 
un paquete completamente inestable.

\subsubsection{Complejidad Ciclomática (Modificabilidad)}
Número de decisiones que pueden ser tomadas en un procedimiento.
Procedimientos con un valor mayor a 15 son difíciles de entender, mientras que con un valor mayor a 30 
son extremadamente complejos y deberían ser divididos en métodos más pequeños (a menos que sea código 
auto-generado).

\subsubsection{índice de mantenibilidad (Modificabilidad)}

Corresponde a un índice entre 0 y 100 que representa la facilitad relativa
de mantener el código. Un valor más alto indica una mejor mantenibilidad. Un valor entre 20 y 100 
indica que el código tiene una buena mantenibilidad. Un valor entre 10 y 19 indica que el código es 
moderadamente mantenible y un código entre 0 y 9 indica una baja mantenibilidad.

\subsubsection{Código duplicado (Reusabilidad)}

Se utiliza alguna heurística para detectar código potencialmente duplicado.
El hecho de encontrar un porcentaje alto de duplicación, podría indicar que no se está haciendo un reuso 
adecuado en el software.

\subsection{Selección de métricas para portabilidad}

\subsubsection{Prácticas de instalación (Instalabilidad)}

Descripciones cualitativas acerca de cómo se implementa una instalación estándar para el producto. 
Se estudian estos procesos y se recomiendan mejoras para alinear las prácticas a estándares profesionales.
%Especificacion para SAE Cliente

\section{Especificación de la evaluación para SAE Framework Cliente}
\subsection{Selección de métricas para mantenibilidad}
\subsubsection{Complejidad Ciclomática (Modificabilidad)}
Número de decisiones que pueden ser tomadas en un procedimiento.
Procedimientos con un valor mayor a 15 son difíciles de entender, mientras que con un valor mayor a 
30 son extremadamente complejos y deberían ser divididos en métodos más pequeños (a menos que sea 
código auto-generado).

\subsubsection{Profundidad (Modificabilidad, Analizabilidad)}
Estudia el nivel de anidamiento que puede existir entre funciones o expresiones dentro del código. 
Se debe definir un límite crítico de profundidad y verificar que no se esté sobrepasando.

\subsubsection{Número de parámetros (Analizabilidad)}
Estudia el número de parámetros en una función.
Al reducir este valor, se puede mejorar la analizabilidad y modularidad del código de manera sustancial.
Al igual que la métrica anterior, se debe definir un valor límite para estudiar el código y verificar que
no se esté sobrepasando.

\subsubsection{Código duplicado (Reusabilidad)}

Se utiliza alguna heurística para detectar código potencialmente duplicado.
El hecho de encontrar un porcentaje alto de duplicación, podría indicar que no se está haciendo un reuso 
adecuado en el software.

%----FIN METRICAS PARA SAE CLIENTE---------------------------------------------------------%
\section{Diseño de la evaluación}
Para llevar a cabo las mediciones, MOSAQ hizo entrega de las fuentes principales de su software, 
los cuales bajo un acuerdo de confidencialidad, fueron analizados por parte del equipo Toeska.

\subsection{Para SAEFramework Servidor}
Para este módulo se utilizaron las siguientes herramientas para realizar las mediciones.

\begin{itemize}
\item NDepend v5.0.0.8085\footnote{http://www.ndepend.com}
\item Visual Studio 2013 Code Metrics\footnote{http://msdn.microsoft.com/en-us/library/bb385914.aspx}
\item Visual Studio 2012 Code Clone Analysis\footnote{http://msdn.microsoft.com/en-us/library/hh205279.aspx}
\end{itemize}

\subsection{Para SAEFramework Cliente}

\begin{itemize}
\item SonarQube\footnote{http://www.sonarqube.org/}
\item JsHint v2.1.11\footnote{http://www.jshint.com/}
\item WebStorm v7 Inspection tools\footnote{http://www.jetbrains.com/webstorm/}


La configuración elegida para JsHint es la siguiente:
\begin{verbatim}
{
    "globals": {
        "console": false,
        "jQuery": false,
        "_": false
    },
    "maxparams": 5,
    "maxdepth": 5,
    "maxstatements": 25,
    "maxcomplexity": 10,
    "es5": true,
    "browser": true,
    "boss": false,
    "curly": false,
    "debug": false,
    "devel": false,
    "eqeqeq": true,
    "evil": true,
    "forin": false,
    "immed": true,
    "laxbreak": false,
    "newcap": true,
    "noarg": true,
    "noempty": false,
    "nonew": false,
    "nomen": false,
    "onevar": true,
    "plusplus": false,
    "regexp": false,
    "undef": true,
    "sub": true,
    "strict": false,
    "white": true,
    "unused": true
}
\end{verbatim}

En esta configuración podemos ver los parámetros \textit{maxparams}, \textit{maxdepth} 
y \textit{maxcomplexity} los cuales nos permiten definir los límites para el número de parámetros, 
profundidad y complejidad ciclomática respectivamente. Estos parámetros fueron seleccionados de manera 
informada de acuerdo a buenas prácticas investigadas previamente y acordes a un software con las 
características de SAE Framework.
\end{itemize}

% PLAN DE ACTIVIDADES DE EVALUACION %
\subsection{Plan de actividades de evaluación}

A través de reuniones con MOSAQ, se acordaron los siguientes puntos dentro del plan de trabajo en lo 
que respecta a planificación y evaluación de calidad.
\begin{itemize}
\item El equipo Toeska realizará las mediciones pertinentes sobre el código fuente y sobre algunas 
prácticas de desarrollo del equipo de MOSAQ. Con estas mediciones se evaluarán los requerimientos de 
calidad establecidos en el modelo que se definió.

\item El equipo Toeska entregará los resultados de la primera evaluación, los puntos donde se encontraron 
brechas que deben ser consideradas, además guías para realizar los ajustes necesarios para que el producto 
obtenga mejores resultados durante la próxima evaluación.

\item Finalmente se realizará otra evaluación utilizando los mismo criterios utilizados 
para la primera evaluación, en la cual se espera haber cerrado las brechas encontradas anteriormente 
y así certificar que el producto cumple con las características escogidas de la norma ISO 25010.
\end{itemize}

% PRIMERA EVALUACION %
\section{Resultados de la primera evaluación SAEFramework Servidor}
\subsection{Realización de mediciones de mantenibilidad}

\subsubsection{Cohesión Relacional}

Los resultados para esta métrica se pueden observar en el cuadro número~\ref{table:CR}.

\begin{table}[hb]
\centering
    \begin{tabular}{|l|l|l|}
    \hline
       \bf{Assemblies}       & \bf{Cohesión Relacional} & \bf{Nombre}           \\ \hline
       SaeFramework2013 & 1.4                   & SaeFramework2013 \\ \hline
    \end{tabular}
    \caption{Cohesión relacional}
    \label{table:CR}
\end{table}


\subsubsection{LCOM}

Los resultados de esta métrica se pueden ver en el cuadro~\ref{table:LOCM}.

\begin{table}[hb]
\centering
    \begin{tabular}{|l|l|}
    \hline
    \bf{Nombre}                                                         & \bf{LCOM}    \\ \hline
       Mosaq.SaeFramework.v2013.Vistas.ListadoAuditoriaTicket      &    0.97 \\ \hline
       Mosaq.SaeFramework.v2013.Negocio.CreacionOSTicket           &    0.82 \\ \hline
       Mosaq.SaeFramework.v2013.Negocio.EdicionTicket              &    0.78 \\ \hline
       Mosaq.SaeFramework.v2013.Datos.Conector                     &    0.68 \\ \hline
       Mosaq.SaeFramework.v2013.Negocio.ManutencionCentroCosto     &    0.5  \\ \hline
       Mosaq.SaeFramework.v2013.Negocio.ManutencionCoberturaSitios &    0.42 \\ \hline
       Mosaq.SaeFramework.v2013.Negocio.ManutencionZonas           &    0.42 \\ \hline
    \end{tabular}
	\caption{Principales tipos y su LCOM}
    \label{table:LOCM}
\end{table}


\subsubsection{LCOM HS}
\begin{table}[hb]
\centering
    \begin{tabular}{|l|l|}
    \hline
    \bf{Nombre}                                                    & \bf{LCOMHS} \\ \hline
       Mosaq.SaeFramework.v2013.Vistas.ListadoAuditoriaTicket      &    1        \\ \hline
       Mosaq.SaeFramework.v2013.Negocio.CreacionOSTicket           &    0.9      \\ \hline
       Mosaq.SaeFramework.v2013.Negocio.EdicionTicket              &    0.88     \\ \hline
       Mosaq.SaeFramework.v2013.Datos.Conector                     &    0.71     \\ \hline
       Mosaq.SaeFramework.v2013.Negocio.ManutencionCentroCosto     &    0.67     \\ \hline
       Mosaq.SaeFramework.v2013.Negocio.ManutencionCoberturaSitios &    0.56     \\ \hline
       Mosaq.SaeFramework.v2013.Negocio.ManutencionZonas           &    0.56     \\ \hline
       Mosaq.SaeFramework.v2013.Utilidades.StringEnum              &    0.5      \\ \hline
    \end{tabular}
    \caption{Principales tipos y su LCOM HS}
    \label{table:LOCMHS}
\end{table}

Los datos del análisis para LCOM HS se observar en el cuadro~\ref{table:LOCMHS}

\subsubsection{Acoplamiento eferente}

\begin{table}[hb]
\centering
    \begin{tabular}{|l|l|}
    \hline
    \bf{Nombre}                                                    & \bf{Tipos que utiliza} \\ \hline
       Mosaq.SaeFramework.v2013.Vistas.ConsultaVistas              &    89                  \\ \hline
       Mosaq.SaeFramework.v2013.Negocio.TicketAcciones             &    34                  \\ \hline
       Mosaq.SaeFramework.v2013.Negocio.ReglasNegocio              &    28                  \\ \hline
       Mosaq.SaeFramework.v2013.Negocio.CreacionOSTicket           &    22                  \\ \hline
       Mosaq.SaeFramework.Utilidades.TypeMap                       &    26                  \\ \hline
       Mosaq.SaeFramework.v2013.Datos.Conector                     &    24                  \\ \hline
       Mosaq.SaeFramework.v2013.Utilidades.Utilidades              &    21                  \\ \hline
       Mosaq.SaeFramework.v2013.Standard.CmdbComponentes           &    20                  \\ \hline
       Mosaq.SaeFramework.v2013.Standard.GruposResolutoresEmpresas &    19                  \\ \hline
       Mosaq.SaeFramework.v2013.Standard.ZonasEmpresas             &    19                  \\ \hline
    \end{tabular}
    \caption{Top 10 de tipos y su acoplamiento eferente}
    \label{table:EC}
\end{table}

Las mediciones para los $10$ tipos con mayor valor de acoplamiento eferente se pueden observar en el cuadro~\ref{table:EC}.

\subsubsection{Acoplamiento aferente}

Las mediciones para los $10$ tipos con mayor valor de acoplamiento aferente se pueden observar en el cuadro~\ref{table:AC}.

\begin{table}[hb]
\centering
    \begin{tabular}{|l|l|}
    \hline
    \bf{Nombre del método}                                                                               & \bf{Métodos que lo utilizan} \\ \hline
       Mosaq.SaeFramework.v2013.Datos.Conector.fabricar()                                                  &    425                         \\ \hline
       Mosaq.SaeFramework.v2013.Datos.IConector.ejecutarProcedimiento                    &    237                         \\ \hline
       Mosaq.SaeFramework.v2013.Datos.IConector.extraerProcedimientoI    &    161                         \\ \hline
       Mosaq.SaeFramework.v2013.Datos.IConector.ejecutarProcedimientoInt                    &    130                         \\ \hline
       Mosaq.SaeFramework.v2013.Datos.IConector.retornaListaI                               &    14                          \\ \hline
       Mosaq.SaeFramework.v2013.Datos.IConector.IniciaTransaccion()                                        &    12                          \\ \hline
       Mosaq.SaeFramework.v2013.Datos.IConector.EjecutaTransaccion()                                       &    12                          \\ \hline
       Mosaq.SaeFramework.v2013.Datos.IConector.CancelaTransaccion()                                       &    12                          \\ \hline
    \end{tabular}
    \caption{Top 10 de métodos y su acoplamiento aferente}
    \label{table:AC}
\end{table}


\subsubsection{Código muerto}
Total: \textbf{581 métodos}.

La lista de métodos que no son llamados se encuentran en una hoja de cálculo adjunta.

\subsubsection{Instabilidad}

Para el cálculo de esta medida se utilizó el promedio del Top 10 de tipos con mayor acoplamiento eferente y el promedio del Top 10 de los tipos con mayor acoplamiento aferente. Estos valores se pueden obtener utilizando los datos de las tablas anteriores. A continuación se presenta el resultado.

$$I = 30.2 / (30.2 + 102.3)$$
$$I = 0.23$$

\subsubsection{Complejidad ciclomática}

En el cuadro~\ref{table:CC} se pueden observar los 10 métodos con mayor complejidad ciclomática.

\begin{table}[hb]
\centering
    \begin{tabular}{|l|l|}
    \hline
    \bf{Nombre}                                                                          & \bf{Complejidad ciclomática} \\ \hline
       Vistas.ConsultaVistas.Lista                              &    77                          \\ \hline
       Negocio.TicketAcciones.Transferir                        &    10                          \\ \hline
       Negocio.TicketAcciones.ReabrirTicket                     &    8                           \\ \hline
       Negocio.TicketAcciones.Recatalogar                       &    5                           \\ \hline
       Negocio.TicketAcciones.ProgramarFechaAtencion            &    4                           \\ \hline
       Negocio.CreacionOSTicket   .grabarYcalcularFechasTickets &    4                           \\ \hline
       Utilidades.Validadores.digitoVerificador                       &    5                           \\ \hline
       Utilidades.StringEnum.Parse                              &    5                           \\ \hline
       Utilidades.Utilidades.IEnumerableToDataTable        &    5                           \\ \hline
       Datos.Conector.ejecutarProcedimientoCadenaAnidada     &    4                           \\ \hline
    \end{tabular}
    \caption{Top 10 de métodos y su complejidad ciclomática}
    \label{table:CC}
\end{table}


\subsubsection{índice de mantenibilidad}

\begin{table}[hb]
\centering
    \begin{tabular}{|l|l|}
    \hline
    \bf{Nombre}      & \bf{índice de mantenibilidad} \\ \hline
    SaeFramework2013 & 93                              \\ \hline
    \end{tabular}
    \caption{índice de mantenibilidad}
    \label{table:MI}
\end{table}

En el cuadro~\ref{table:MI} se muestra el valor del índice de manteniblidad para el código de software.

\newpage

\subsubsection{Código duplicado}

En el cuadro~\ref{table:clones} se pueden observar los clones de código 
encontrados en el software. En la primera columna se presenta el grupo y la 
intensidad de la duplicación encontrada, mientras que en la segunda columna el
nombre de los archivos y las líneas donde se encontró el código duplicado.

\begin{table}[hb]
\centering

  \begin{tabular}{ | l | l | }
  
  \hline
    \bf{Grupo de Clones} & \bf{Nombre} \\ \hline
    Match Fuerte (1 archivo) & Negocio\textbackslash TicketAcciones.cs líneas 75-94 \\ \hline
     & Negocio\textbackslash TicketAcciones.cs líneas 360-380 \\ \hline
    Match Medio (2 archivos) & Negocio\textbackslash ManutencionCoberturaSitios.cs líneas 69-94 \\ \hline
     & Negocio\textbackslash ManutencionZonas.cs líneas 69-94 \\ \hline
    Match Débil (1 archivo) & Negocio\textbackslash TicketAcciones.cs líneas 119-143 \\ \hline
     & Negocio\textbackslash TicketAcciones.cs líneas 304-329 \\ \hline
     & Negocio\textbackslash TicketAcciones.cs líneas 207-231 \\ \hline
     & Negocio\textbackslash TicketAcciones.cs líneas 43-65 \\ \hline
    Match Débil (2 archivos) & Negocio\textbackslash ManutencionCoberturaSitios.cs líneas 103-127 \\ \hline
     & Negocio\textbackslash ManutencionZonas.cs líneas 104-128 \\ \hline
  \end{tabular}
  \caption{Clones de código}
  \label{table:clones}
\end{table}




\subsection{Realización de mediciones de portabilidad}

\subsubsection{Observaciones}
\begin{itemize}
\item Se debe obtener documentación necesaria por parte de MOSAQ para analizar 
las prácticas habituales con respecto a temas de instalabilidad y así proponer recomendaciones.
\end{itemize}
%DESCRIPCIONES CUALITATIVAS DE PROCESOS DE INSTALABILIDAD%

%---------------------------------------RESULTADOS PARA SAE CLIENTE -----------------------------------------%
\section{Resultados de la primera evaluación SAEFramework Cliente}

Para el estudio de este módulo se evaluaron los siguientes directorios y archivos:
\begin{itemize}
\item Js/app.js
\item Js/BasePrincipal.js
\item Js/controllers
\item Js/app/connection/wcf.js
\item Js/app/services/services.js
\end{itemize}


\subsection{Realización de mediciones para mantenibilidad}
\subsubsection{Complejidad Ciclomática}

En el cuadro~\ref{table:JSCC} se pueden observar los archivos y directorios 
seleccionados para la evaluación, así como su respectiva complejidad por método.
\begin{table}[hb]
\centering
    \begin{tabular}{|l|l|}
    \hline
    \textbf{Archivo/Directorio}            & \textbf{Complejidad por método} \\ \hline
    app.js                        & 4.1                    \\ \hline
    BasePrincipal.js              & 3.0                    \\ \hline
    app/controllers/debug         & 1.2                    \\ \hline
    app/controllers/admin         & 1.7                    \\ \hline
    app/controllers/laboratorio   & 1.0                    \\ \hline
    app/controllers/main          & 1.9                    \\ \hline
    app/controllers/ordenservicio & 1.7                    \\ \hline
    app/controllers/tickets       & 2.2                    \\ \hline
    app/connection/wcf.js         & 4.1                    \\ \hline
    app/services/services.js      & 1.8                    \\ \hline
    \end{tabular}
    \caption{Elementos elegidos y su complejidad por método}
    \label{table:JSCC}
\end{table}

\subsubsection{Profundidad}
En el cuadro~\ref{table:depth} se pueden observar los archivos y directorios 
seleccionados para la evaluación y su análisis respectivo de profundidad. Se 
debe recordar que se eligió utilizar un valor de 5 como mínimo para que el 
elemento comience a ser crítico.
\begin{table}[hb]
\centering
    \begin{tabular}{|l|l|}
    \hline
    \textbf{Archivo/Directorio}            & \textbf{Profundidad}                 \\ \hline
    app.js                        & Sin profundidad mayor que 5 \\ \hline
    BasePrincipal.js              & Sin profundidad mayor que 5 \\ \hline
    app/controllers/debug         & Sin profundidad mayor que 5 \\ \hline
    app/controllers/admin         & Sin profundidad mayor que 5 \\ \hline
    app/controllers/laboratorio   & Sin profundidad mayor que 5 \\ \hline
    app/controllers/main          & Sin profundidad mayor que 5 \\ \hline
    app/controllers/ordenservicio & Sin profundidad mayor que 5 \\ \hline
    app/controllers/tickets       & Sin profundidad mayor que 5 \\ \hline
    app/connection/wcf.js         & Sin profundidad mayor que 5 \\ \hline
    app/services/services.js      & Sin profundidad mayor que 5 \\ \hline
    \end{tabular}
    \caption{Elementos elegidos y el match para profundidad seleccionado}
    \label{table:depth}
\end{table}
\subsubsection{Número de parámetros}

En el cuadro~\ref{table:PN} se pueden observar los archivos y directorios 
seleccionados para la evaluación y su análisis respectivo del número de 
parámetros en sus funciones. Para esta métrica se decidió utilizar un mínimo 
de 5 parámetros para considerar un elemento como peligroso.
\begin{table}[hb]
\centering
    \begin{tabular}{|l|l|}
    \hline
    Archivo/Directorio            & Número de parámetros \\ \hline
    app.js                        & No sobrepasa             \\ \hline
    BasePrincipal.js              & No sobrepasa             \\ \hline
    app/controllers/debug         & No sobrepasa             \\ \hline
    app/controllers/admin         & No sobrepasa             \\ \hline
    app/controllers/laboratorio   & No sobrepasa             \\ \hline
    app/controllers/main          & No sobrepasa             \\ \hline
    app/controllers/ordenservicio & No sobrepasa             \\ \hline
    app/controllers/tickets       & No sobrepasa             \\ \hline
    app/connection/wcf.js         & No sobrepasa             \\ \hline
    app/services/services.js      & No sobrepasa             \\ \hline
    \end{tabular}
    \caption{Elementos elegidos y el match para número de parámetros seleccionado}
    \label{table:PN}
\end{table}
\subsubsection{Código duplicado}

No se han encontrado registros de código duplicado relevante. El análisis 
sólo entregó duplicación con respecto a la versión minificada del software, 
lo cual no corresponde a duplicación que afecte la mantenibilidad.

\section{Conclusiones SAEFramework Servidor}

\subsection{Mantenibilidad}

El primer valor analizado es el de Cohesión relacional. El valor para este análisis es de 1.4. Este resultado se
encuentra dentro del rango considerado como correcto para una aplicación. Valores
más altos podrían indicar sobre-acoplamiento ya que la cohesión relacional nos entrega un promedio de relaciones internas por tipo dentro de un
paquete. Un paquete debe tener sus clases fuertemente relacionadas y estas no deberían estar relacionadas de manera
considerable con clases fuera de este paquete.
\\

Las siguientes métricas corresponden a LCOM y LCOM HS. Para estas medidas existen rangos definidos que son considerados como correctos para un conjunto de tipos.

Con respecto a LCOM, se deben analizar los tipos que tengan un valor mayor a 0.8. En este caso existen 2 tipos que sobrepasan ese valor. Estos tipos son:
\begin{itemize}
\item CreacionOSTicket (Clase)
\item ListadoAuditoriaTicket (Clase)	
\end{itemize}

Se han analizado estas clases y se concluyó que no es necesario realizar 
cambios sobre ellas ya que sólo están compuestas por \textit{getters} 
y \textit{setters}. Las clases de este estilo no son correctamente evaluadas 
por esta métrica ya que no tienen un comportamiento definido por relaciones 
entre sus métodos.
\\

Para los valores de LCOM HS se recomienda analizar los tipos cuyos resultados 
sean estrictamente mayores que 1. En las mediciones no se encontraron tipo 
que sobrepasen este umbral. Solo la clase \textbf{ListadoAuditoriaTicket} 
arrojó un valor igual a 1, sin embargo por las razones explicadas en el análisis 
de LCOM, esta clase no está mal implementada de acuerdo a esta métrica.
\\

El valor de \textbf{acoplamiento eferente} podría revelar tipos que tienen 
muchas responsabilidades. Mientras mayor sea este valor, más entrelazado está 
el tipo con otras implementaciones. Si bien los valores obtenidos no son altos, 
se recomienda analizar los primeros tipos y verificar si pueden ser más modularizados.
\\

Para la métrica de \textbf{acoplamiento aferente} se pueden observar el top 10 
de los tipos con mayores llamadas dentro del sistema. Estos valores no indican 
problemas de diseño y sirven para estudiar cuales son los tipos más utilizados. 
Esta información se entrega con el fín de obtener un estudio más detallado 
acerca del software.
\\

Con respecto al \textbf{código muerto}, se entrega una planilla con los métodos 
para los cuales se encontraron 0 llamadas. Se recomienda analizar estos métodos 
y verificar que realmente podrían ser necesarios en el sistema o de lo contrario 
eliminarlos ya que pueden degradar la analizabilidad del software.
\\

El valor de \textbf{instabilidad} es de 0.23. Este valor puede variar entre 
 y 1, donde 0 indica un paquete completamente estable y 1 un paquete 
completamente inestable. El valor obtenido en este caso es bastante aceptable 
e indica que el software se encuentra en un estado estable y tiene una buena 
modificabilidad. Cabe mencionar que este valor se obtiene analizando las 
métricas de acoplamiento descritas anteriormente, de esta manera este valor 
esta relacionado con el nivel de acoplamiento en el sistema.
\\

Para estudiar la \textbf{complejidad ciclomática} se muestra una tabla con 
el top 10 de los métodos con la mayor complejidad. Los estándares para los 
métodos indican que un método con una complejidad ciclomática mayor a 30 puede 
ser demasiado complejos y se debe estudiar la posibilidad de dividirlo en 
métodos más pequeños a menos que corresponda a código generado.
En este caso el análisis arrojó solo un método con un valor superior a 30, 
el cual es \textbf{Listar()} dentro de \textbf{Mosaq.SaeFramework.v2013.Vistas.ConsultaVistas}.
Se analizó este método y se puede observar de que contiene un \textit{switch} 
con un gran número de \textit{cases}. Este \textit{switch} es el que genera 
un aumento en la complejidad ciclomática. Por lo tanto se recomienda estudiar 
este método y buscar alguna manera de refactorizarlo y así obtener un código 
más mantenible. Algunas opciones para refactorizar un \textit{switch} 
se basan en utilizar diccionarios, mapas o simplemente separar las enumeraciones 
en sus propias clases.
\\

El valor obtenido para el \textbf{índice de mantenibilidad} es de 93. Este 
valor puede estar entre 0 y 100 y mientras más alto indica una mejor 
mantenibilidad. En este caso el valor sugiere que el software es altamente 
mantenible bajo los estándares de esta métrica.
\\

Finalmente, el análisis de código duplicado entrega una tabla en la cual se 
pueden observar archivos que contienen posibles fuentes de código duplicado. 
Si bien este análisis muestra que el código duplicado es mínimo, se recomienda 
estudiar estos archivos y ver si se puede refactorizar código para así disminuir 
aún más la cantidad de duplicación.



%-----------------------------CONCLUSIONES CLIENTE-----------------------------------------------------%


\section{Conclusiones SAEFramework Cliente}

\subsection{Mantenibilidad}

Para este módulo no se encontraron deficiencias relevantes con respecto a la 
mantenibilidad del código.

Para realizar el análisis se debe tener en cuenta que la arquitectura para 
esta parte del framework difiere profundamente con la arquitectura servidor. 
Esto es debido al uso de herramientas nuevas que utilizan javascript a través 
de patrones de diseño que clásicamente no se aplicaban en ingeniería de \textit{front-end}.
\\

Para el estudio de complejidad ciclomática se debe estudiar la complejidad por 
método y no por archivo, puesto que en esta arquitectura un archivo no 
necesariamente representa a una clase y usualmente define una serie de funciones 
asíncronas definidas para su framework. Es por esto que la complejidad 
ciclomática de un archivo puede resultar alta, sin embargo por método es baja. 
Luego de estudiar los archivos y directorio elegidos, no se encontraron métodos 
con complejidad alta por lo cual esta parte del código tiene buena mantenibilidad 
con respecto a sus complejidad.
\\

Al encontrar funciones con un valor de profundidad muy alto, se generan 
problemas de analizabilidad y por ende de mantenibilidad. Para este caso, 
como es un framework de mediana magnitud, se decidió utilizar un valor de 5 
anidaciones para no ser demasiado restrictivo, ni demasiado permisivo. 
Al realizar el análisis no se encontró ningún elemento que sobrepase este 
umbral, por lo cual esta característica no está afectando la analizabilidad del 
sistema.
\\

EL número de parámetros que un método recibe también es una métrica que se debe 
tener en cuenta para estudiar la analizabilidad de un sistema. Mientras más 
parámetros recibe una función, más compleja se vuelve su analizabilidad y 
mantenibilidad. En este caso utilizando los mismo argumentos que en la métrica 
anterior, se escogió un número de 5 parámetros como mínimo para considerar a 
una función como crítica. Luego de realizar el análisis, no se encontraron 
funciones que sobrepasen este umbral, por lo cual esta es otra característica 
bien implementada en el sistema y que mejora su mantenibilidad.
\\

Finalmente como se mencionó en la sección anterior, el software que se utilizó 
para realizar las mediciones de duplicación, sólo entregó duplicaciones de 
varios archivos con respecto a la versión minificada del framework, por ende 
no son consideradas.

\section{Recomendaciones}

Para ambos módulos se recomienda el uso de alguna herramienta de análisis 
de código estático. Estas herramientas permiten estudiar algunas de las métricas 
presentadas en este informe así cómo buenas prácticas correspondientes al 
lenguaje de la aplicación.

Los valores adecuados para cada métrica se pueden encontrar en la 
sección~\ref{section:specification}. 

Para el código de SAE Servidor se pueden utilizar las métricas de código 
ofrecidas por Visual Studio. A través de estas métricas se puede analizar el 
índice de mantenibilidad con el fin de que no se escape de los márgenes 
apropiados. Este análisis debe ser complementado con alguna otra herramienta 
de análisis estático. Se estudiaron 2 herramientas las cuales tienen ciertas 
ventajas y desventajas. 

La primera herramienta es NDepend, la cual se utilizó para obtener la mayoría 
de las métricas que se presentaron en la sección de análisis de SAE Servidor. 
Esta herramienta es altamente profesional, se integra de manera perfecta con 
Visual Studio y entrega reportes muy detallados. Este software es propietario 
y tiene un costo asociado.

Por otro lado se analizó la herramienta SonarQube, la cual se utilizó para 
obtener la mayoría de medidas para SAE Cliente. Este software también tiene 
una extensión para .NET lo cual permitió realizar pruebas sobre el código 
fuente del lado servidor.
SonarQube presenta buenos resultados, los cuales pueden ser útiles para 
perpetuar la mantenibilidad del código. Además, esta herramienta es de código 
abierto lo cual la ubica como la mejor opción puesto que no se desea hace un 
análisis demasiado exhaustivo sino más bien encontrar puntos críticos dentro 
del producto que puedan afecta su mantenibilidad. El uso de Sonarqube también 
se recomienda para analizar el código del lado cliente.



\section{Evaluación final para SAEFramework Servidor}
La evaluación final se realizó utilizando la versión más reciente del producto 
SAE Framework Servidor. Para esta evaluación se incluyen además observaciones 
con respecto a otras características no incluidas en la primera evaluación y 
que corresponden a Instalabilidad y la sub-característica capacidad de pruebas. 
Esta última comenzó a ser implementada durante el transcurso de la evaluación
por lo que inicialmente no se contaba con pruebas de software.
A continuación se presenta una síntesis de las métricas más relevantes para 
este punto de la evaluación.

\subsection{Mediciones de mantenibilidad}

\subsubsection{Cohesión relacional}
Se obtuvo un valor similar al de la primera evaluación.
\subsubsection{LCOM}
Se obtuvieron valores similares a los de la primera evaluación para el Top 10 de métodos
con mayor valor para esta métrica.
\subsubsection{LCOM HS}
Se obtuvieron valores similares a los de la primera evaluación para el Top 10 de métodos
con mayor valor para esta métrica.
\subsubsection{Acoplamiento Eferente}
Se obtuvieron similares a la evaluación anterior, sin embargo esta vez no aparece en primer lugar
el método ConsultaVistas(), el cual sobrepasaba el umbral de acoplamiento recomendable.
\subsubsection{Acoplamiento Aferente}
Se obtuvieron similares a la evaluación anterior.
\subsubsection{Instabilidad}
Nuevo valor obtenido: 0.18.
Este valor descendió debido a que el acoplamiento eferente ya no aumenta 
demasiado su valor, gracias a la refactorización del método con mayor valor obtenido en la primera evaluación.
Cabe recordar que mientras más bajo sea este valor, más mantenible es el producto de software.
\subsubsection{Complejidad Ciclomática}
Al igual que en acoplamiento eferente, el método ConsultarVista() ya no lidera el top 10 de métodos
con mayor complejidad. Este método agregaba un valor de complejidad 
ciclomática de 77 en la primera evaluación, el cual ya no aparece en el ranking. 
De esta manera el método con mayor complejidad tiene un valor de 10 y 
corresponde al segundo en la lista de la primera evaluación. El resto de la lista
es similar.
\subsubsection{índice de Mantenibilidad}
El índice de mantenibilidad entregado por la herramienta \textit{Code metrics} de Visual Studio 2012
aún entrega un excelente valor de 93.
\subsubsection{Código Duplicado}
El análisis de \textit{Code Clones} provisto por Visual Studio 2012 entregó matches para los 
mismos archivos descritos en la primera evaluación.
\subsection{Capacidad de Pruebas}
\label{cap}

La capacidad para realizar pruebas en el software es fundamental para un producto mantenible y es por eso
que MOSAQ ha decidido implementar esta propuesta.

Inicialmente se tenía un 0\% de cobertura de pruebas y durante el desarrollo de la evaluación, el equipo Toeska entregó un plan para comenzar a implementar
módulos de pruebas y un ambiente de testing para el equipo de desarrollo.

Se clasificaron diferentes servicios y sus respectivas pruebas de acuerdo a su criticidad.
En el cuadro~\ref{table:CBPC} se puede observar el total de pruebas para cada 
categoría, mientras que en el cuadro~\ref{table:CBC} se presenta el cumplimiento 
para estas mismas de acuerdo a los rangos recomendados por el equipo Toeska.

Con esta información podemos concluir que esta característica se cumple correctamente.
\begin{table}[hb]
\centering
    \begin{tabular}{lll}
    ~               & Total  & \%        \\
     		Críticas:    &  		12  &  		8.51   \\
     		Importantes: &  		28  &  		19.86  \\
     		Deseables:   &  		101 &  		71.63  \\
     		Todas        &  		141 &  		100.00 \\ \hline
    \end{tabular}
     \caption{Cobertura de pruebas - porcentaje por criticidad}
    \label{table:CBPC}
\end{table}

\begin{table}[hb]
\centering
    \begin{tabular}{llllll}
    ~               &  		Total &  		Faltan &  		\% Real &  		\% Meta & Cumple? \\
     		Críticas:    &  		12    & 0         &  		100.00 &  		100.00 &  		SI    \\
     		Importantes: &  		28    & 0         &  		100.00 &  		80.00  &  		SI    \\
     		Deseables:   &  		101   & 0         &  		100.00 &  		40.00  &  		SI    \\
     		Todas        &  		141   & ~         & ~         & ~         & ~        \\ \hline
    \end{tabular}
         \caption{Cobertura de pruebas - Cumplimiento}
    \label{table:CBC}
\end{table}

\subsection{Portabilidad (Instalabilidad)}
\label{ins}
Para esta etapa de la evaluación se decidió evaluar la Instalabilidad del software, la cual pertenece
a la característica Portabilidad del modelo ISO/IEC 25010. Esta característica toma fuerte relevancia
en el caso de que terceros deban mantener el software a futuro.

Para realizar un análisis de esta característica se estudió un documento provisto por MOSAQ, el cual
muestra la información relevante con los procesos involucrados en la instalación del software.
En este documento se especifica entre otras cosas:

\begin{itemize}
\item Los pre-requisitos de Hardware y Software que el cliente debe tener para poder llevar a cabo la instalación correcta del programa.
\item Descripción de los componentes a ser instalados.
\item Consideraciones de arquitectura en el caso de elegir un servicio simple o distribuido.
\item Descripción detallada de los pasos a seguir durante el Setup del programa.
\item Configuraciones adicionales que requiere el servicio.
\item Descripción de los pasos a seguir durante la instalación del componente web de la aplicación.
\end{itemize}

La descripción detallada de los procesos necesarios para la instalación 
se encuentran con un buen nivel de completitud, por lo cual se cumple con un 
buen estándar para esta sub-característica.

\section{Evaluación final para SAEFramework Cliente}
\subsection{Evaluación final de mantenibilidad}
\subsubsection{Complejidad Ciclomática}
El análisis arrojó resultados similares a los de la primera evaluación. Algunos archivos aumentaron
de forma mínima su complejidad. Por ejemplo app.js aumentó de 4.1 a 4.3.
Los valores en general de los archivos siguen siendo muy buenos.
Se realizó un barrido buscando cualquier archivo que mostrara una complejidad por función mayor
a 5 y sólo se encontró el archivo util/sae-message.js con una complejidad de 5.1 lo cual aún es
un valor aceptable.
Se puede concluir que el nivel de complejidad del sistema en general se encuentra dentro
de los rangos requeridos para un software mantenible.
\subsubsection{Profundidad}
Al igual que en la primera evaluación, ningún archivo sobrepasa la profundidad de anidamiento
recomendada.
\subsubsection{Número de parámetros}
No se encontró ningún archivo que sobrepase el número de parámetros recomendado para
esta evaluación.
\subsubsection{Código duplicado}
No existe duplicación relevante para los archivos puestos en evaluación. Sin embargo se ejecutó
un análisis sobre el proyecto completo y se encontró un porcentaje de duplicación de un 7.7\%.
Si bien este porcentaje es bajo, se recalca el uso de alguna herramienta de análisis estático para
refactorizar estas duplicaciones y para realizar un estudio general de la calidad del código en el futuro.
\subsubsection{Capacidad de pruebas}

Esta sub-característica se estudió en conjunto con la parte SAE Framework 
Servidor. La información relevante se puede encontrar la sección~\ref{cap}.

\subsection{Evaluación final de Portabilidad (Instalabilidad)}

Para esta subcaracterística se indicó el contenido principal de la documentación de instalación
del software SAE en la sección de SAEFramework servidor. Como se mencionó en ese punto, existe
documentación detallada para llevar a cabo la instalación de la parte 
servidor como para la parte cliente además de otras consideraciones a tener en cuenta durante ese proceso.
Se concluye que el producto cuenta con un buen respaldo y se adhiere 
a estándares acordes con respecto a esta característica. Mas información se puede encontrar en la sección~\ref{ins}

\section{Conclusiones finales de la evaluación}
Durante el proceso de certificación se adaptó un subconjunto de
elementos del modelo ISO/IEC 25010 con el fin de evaluar continuidad de 
desarrollo del producto SAE para la empresa MOSAQ.
Esta continuidad se encuentra fuertemente ligada a conceptos tales como 
mantenibilidad, calidad de documentación y calidad en la gestión de configuración.

El modelo generado se utilizó para realizar dos evaluaciones y elaborar un plan de acción para mejorar el nivel de calidad del producto objetivo.

Los resultados de la evaluación final muestran que el estado de calidad del 
software alcanza un alto nivel y cumple con los estándares necesarios para 
certificar el producto de acuerdo al modelo. Esta certificación incluye 
características pertenecientes al modelo ISO/IEC 25010 así como el plan de 
acción para mejorar la capacidad de pruebas, el control de versiones y la documentación.
