\chapter{Problema}
\label{chap:problema}
El problema principal consiste en la falta de un modelo de calidad apropiado
que permita evaluar la continuidad de desarrollo de un producto de software.
Este modelo debe estar enfocado en los aspectos que más contribuyan a la mantenibilidad
del producto, ya que, como se mencionó previamente, esta es la característica que más
influye en la continuidad de desarrollo.
Cuando este desarrollo es tomado por terceros, la mantenibilidad
del software se torna crítica, ya que afecta directamente a todos los procesos y tareas
que este desarrollo conlleva, ya sea la mantención, eliminación de fallas, análisis de código,
extensión del software, reutilización de módulos o funciones, etc.

El problema secundario aparece una vez definido el modelo principal, con sus características
y subcaracterísticas. Para realizar una evaluación sobre un producto de software real, se deben
utilizar métricas que representen y den un valor medible a estas subcaracterísticas.
Esta elección debe ser apropiada y acorde a las prácticas y estándares utilizados en la industria
hoy en día.

Además de las métricas, se debe diseñar un proceso de evaluación que sirva como guía a los
evaluadores y que permita realizar las tareas de evaluación de manera sistemática y organizada,
siempre tomando en cuenta el contexto del problema.

Teniendo estos componentes: el modelo de calidad con sus características y subcaracterísticas,
la métricas que dan valor medible a estas subcaracterísticas, y las tareas correspondientes al proceso de evaluación, 
se puede llevar a cabo una evaluación exitosa sobre el producto de software.
