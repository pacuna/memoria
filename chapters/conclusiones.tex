\chapter{Conclusiones}
\label{chap:conclusiones}

En este trabajo se ha presentado un modelo de calidad enfocado en la continuidad
de desarrollo para un producto de software. Se ha argumentado porqué la principal característica
que influye en esta continuidad de desarrollo es la mantenibilidad.
Para generar un modelo enfocado esta característica, se ha tomado un conjunto
de subcaracterísticas del estándar ISO 25010, el cual entrega un modelo para
evaluar productos de software. El modelo generado ha sido
refinado para realizar evaluaciones de mantenibilidad. Se han
definido distintas métricas para cada subcaracterística y además se entregan
descripciones de las tareas requeridas para llevar a cabo el proceso de evaluación
sobre un software real de manera efectiva.

Se ha demostrado el uso de este modelo sobre un producto real. A través de esta
evaluación se puede ejemplificar el uso de las guías para el proceso de evaluación
así como el uso de las métricas para evaluar el código fuente del software.
Este ejemplo sirve además para indicar la importancia que tiene el definir correctamente
el propósito y los requerimientos de la evaluación, puesto que aplicar
ciegamente un modelo sobre un producto no es correcto. Cada evaluación tendrá
aspectos relevantes e irrelevantes, por lo que el modelo, que fue acotado
para evaluar mantenibilidad, debe ser aún mayormente acotado para ajustarse
al contexto del producto. En la aplicación presentada en este trabajo, se pueden
notar las diferencias al evaluar la parte cliente con la parte servidor del producto.
Este tipo de diferencias forzará al evaluador a replantearse cada una de las métricas
propuestas para ver cuales tienen sentido en cada módulo que se desee evaluar.

Finalmente, cabe mencionar que realizar una evaluación de calidad, no es algo trivial.
Se requiere más que un modelo y un conjunto de métricas. El evaluador debe ser
capaz de entender el producto de sofware, conocer los requerimientos de usuario,
conocer las tecnologías empleadas, conocer las buenas prácticas para este tipo
de tecnologías y además estar al tanto de las tendencias actuales con respecto
a la mantenibilidad del software. Por ejemplo, para la subcaracterística
que evalúa la capacidad de pruebas de un producto, tenemos diversos tipos
de pruebas que se pueden aplicar. Existen las pruebas unitarias, de integración,
funcionales, de aceptación, regresión, etc. Además existen diversos esquemas
que el equipo de desarrollo puede utilizar para guiar la construcción de sus pruebas.
Con todos estos elementos, el evaluador debe ser capaz de elegir la aplicación
que mejor se adapte al contexto según su criterio, más aún si se creará un plan
de calidad para eliminar brechas, como se realizó en la aplicación de este trabajo.

