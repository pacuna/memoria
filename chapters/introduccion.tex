\chapter{Introducción}
En este trabajo se propone un modelo de calidad para evaluar la continuidad
de desarrollo de un producto de software, así como las tareas necesarias
para llevar a cabo una evaluación de calidad.
La base de este modelo se encuentra en la serie ISO 25000
específicamente en la subdivisión de modelo de calidad.
Se han realizado ajustes para adaptar y refinar el modelo con el fin de evaluar la mantenibilidad de un producto de software.
Esta característica influye directamente en la continuidad de desarrollo y puede
ser descrita a través subcaracterísticas y métricas que sean capaces de cuantificarlas.

Se presenta también una aplicación del modelo generado, sobre un producto de software
real, con el fin de mostrar los pasos principales a la hora de llevar a cabo
una evaluación de calidad y para ejemplificar el uso del framework propuesto. Este se compone
del modelo de mantenibilidad, de las métricas respectivas y de las tareas principales tras el proceso de evaluación.

En el capítulo~\ref{chap:contexto} se introducen los principales conceptos trabajados
en esta memoria y el contexto sobre el cuan han sido abordados.

El problema principal es presentado en el capítulo~\ref{chap:problema} así como
los problemas secundarios que se ramifican de este. El estado del arte relacionado
con el problema se puede encontrar en el capítulo~\ref{chap:arte}

La propuesta de modelo de mantenibilidad así como la descripción de las tareas del proceso
de evaluación son descritos en el capítulo~\ref{chap:propuesta}. En esta sección
también se mencionan las principales características del modelo base del cual
se refinó el modelo de mantenibilidad.

La validación del modelo se puede encontrar en el capítulo~\ref{chap:validacion}.
Para esta sección se realizó una evaluación de calidad sobre un producto de software
real. Se describen las tareas del proceso de evaluación, así como los resultados
de la aplicación del modelo generado con el fín de certificar que el software
cumple con el criterio de evaluación definido.

Finalmente las conclusiones del trabajo son descritas en el capítulo~\ref{chap:conclusiones}.
