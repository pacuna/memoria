%hablar sobre calidad, mantenibilidad y su principal función en el desarrollo.

\chapter{Contexto}
\label{chap:contexto}
%sobre calidad
\section{Calidad de Software}
Un tema de gran importancia en los procesos de desarrollo de software es la calidad.
Sabemos que un producto de software que no cuenta con los estándares de
calidad adecuados, no gozará de propiedades tales como mantenibilidad, seguridad, 
usabilidad, etc.

Si bien la calidad es un concepto subjetivo, se han desarrollado un gran
número de intentos para definir una base común sobre la cual se pueda evaluar un producto 
de software o que sirva para definir un proceso de desarrollo apropiado con el fin de construir
un producto confiable. Estos intentos generalmente apuntan a definir un modelo
de calidad que sea capaz de agrupar las principales características que un producto debe poseer.

Las organizaciones ISO\footnote{http://www.iso.org}, IEC\footnote{http://www.iec.ch} y IEEE\footnote{http://www.ieee.org} 
definen calidad a través de 6 distintas alternativas~\cite{5276043}:
\begin{enumerate}
    \item El grado en el cual un sistema, componente o proceso cumple con los requerimientos especificados.
    \item La capacidad de un producto, servicio, sistema o proceso para cumplir con las necesidades, expectativas
    o requerimientos del usuario o cliente.
    \item El conjunto de características de una entidad que le confieren su habilidad de satisfacer los requerimientos
    declarados y además los implícitos.
    \item Conformidad en las expectativas del usuario, conformidad en los requerimientos del usuario, satisfacción del cliente,
    confiabilidad y el nivel presente de defectos.
    \item El grado en el cual un conjunto inherente de características cumple con los requerimientos.
    \item El grado en el cual un sistema, componente o proceso cumple con las necesidades o expectativas de un cliente o usuario.
\end{enumerate}

Como se puede observar en las definiciones, calidad no tiene una definición universal e incluso se considera
un concepto de carácter filosófico~\footnote{\url{http://en.wikipedia.org/wiki/Quality_(philosophy)}}.

Un modelo clásico de calidad de producto, que puede ser aplicado en la industria
de software es el de Garvin~\cite{Garvin:1984}. Garvin presenta los siguientes
enfoques que se pueden utilizar para evaluar calidad:
\begin{itemize}
    \item Enfoque trascendente
    \item Enfoque basado en el producto
    \item Enfoque basado en el usuario
    \item Enfoque de fabricación
    \item Enfoque basado en valor
\end{itemize}

El \textbf{enfoque trascendente} es el más difuso. Se refiere a la propiedad inherente
e indefinible de un producto. Podría decirse que es casi una característica 
intuitiva con la cual se sabe si un producto es o no de calidad.

El \textbf{enfoque basado en producto} describe diferencias en la medida de algunos
atributos deseados en el producto. Por lo tanto a diferencia del enfoque
trascendente, este puede ser medido. Asumimos que se conoce y se puede 
describir lo que se desea medir.
Este enfoque es complicado en productos de software puesto que algunas métricas
pueden o no existir o ser muy difíciles de cuantificar. Un ejemplo es la mantenibilidad.
Si se asocia mantenibilidad al esfuerzo requerido para completar un cambio, este
esfuerzo es difícil de medir a través de diversos desarrolladores ya que no están
constantemente midiendo el tiempo en sus respectivas tareas. Además depende de 
otros factores como la complejidad del cambio.

En el \textbf{enfoque basado en usuario}, se asume que el producto que satisface
las necesidades del usuario de manera más integra, es el que tiene la mayor calidad. El énfasis
no está en los requerimientos, sino en la impresión subjetiva de los usuarios.

Una visión más interna se observa en el \textbf{enfoque de fabricación}. En este enfoque
calidad se define como la conformidad con respecto a los requerimientos especificados.
Se debe asumir que siempre será posible definir un requerimiento así como la desviación
del producto real con respecto a este. En este enfoque las métricas
concretas, por ejemplo defectos por línea de código son mayormente útiles mientras puedan
relacionarse con algún requerimiento especificado.

Finalmente en el \textbf{enfoque basado en valor} se asignan costos con respecto a la conformidad y no
a los requerimientos. Estos costos se comparan con los beneficios del
producto y con estos datos se calcula el valor.

Garvin sugiere que estos distintos enfoques pueden ser útiles durante distintas
etapas del ciclo de vida del producto. Por ejemplo, al inicio del ciclo de 
vida, durante la incepción del producto, el enfoque debe estar en el usuario
y en el valor, con el fin de entender qué es lo que tiene más valor para
los clientes y usuarios. O por ejemplo, cuando se está construyendo el producto,
el enfoque debería estar en la fabricación, con el fin de asegurar que se está
construyendo un producto adecuado con respecto a la especificación.

\subsection{Calidad de procesos}
Otro elemento importante es la calidad de procesos. La principal idea detrás de este enfoque, es que mientras de más
calidad sean los procesos, de más calidad serán los productos.

Un estándar utilizado frecuentemente en la calidad de procesos es la ISO 9000. 
La idea detrás de este estándar consiste en establecer un sistema de gestión de calidad de
manera que la calidad del producto resultante sea alta. No tiene que ver
directamente con la calidad del producto, sino con los requerimientos de calidad
de la compañía que produce el producto.
La introducción de este tipo de estándares dentro de una compañía de software
puede ser beneficiosa al hacer que los procesos de aseguramiento de calidad sean
más explícitos y claros~\cite{Wagner:2013}.

Otras iniciativas similares, que se enfocan específicamente en mejorar los
procesos de desarrollo de software son CMMI\footnote{http://www.sei.cmu.edu/cmmi} y SPICE (ISO/IEC 15504).
Estos estándares se basan en la premisa de que existe un proceso ideal,
el cual es descrito en los estándares y que cada compañía debe alcanzar.
Existen niveles de madurez y de capacidad partiendo desde procesos caóticos,
hasta procesos altamente estandarizados y optimizados.

La utilización de este enfoque para evaluar continuidad de desarrollo sufre
de ciertas carencias. Para empezar se parte con la premisa de que mientras
mejores sean los procesos, mejor será la calidad del producto. Esto quizás
es claro en fabricación de productos típicos, pero no así en desarrollo de
software. Por ejemplo en~\cite{Jones:2000:SAB:335582} se presenta una 
investigación que buscaba la relación entre el nivel CMM (predecesor de CMMI) 
de una compañía, y el número de defectos por punto de función entregado en sus productos.
Como resultado se encontró que mientras el nivel aumentaba, los errores disminuían,
lo cual confirmaría la premisa. Sin embargo en los extremos, las mejores
compañías que estaban en nivel CMM 1 (el peor), producían software con menos
defectos que las peores compañías con CMM 5 (el mejor). Por lo tanto
habían otros factores que estaban influyendo en la tasa de defectos.

Es importante analizar la calidad de procesos para lograr entregar un producto
de software de alta calidad, sin embargo como muchos otros factores influyen
en la industria del software, este trabajo está más enfocado en estudiar
la calidad del producto, y a través de esta calidad encontrar un modelo que
permita evaluar también la continuidad de desarrollo.

\section{Mantenibilidad}

%sobre mantenibilidad
La mantenibilidad es un tema recurrente en la ingeniería de software y es la característica que
está mayormente ligada a la continuidad de desarrollo de un producto.
Al igual que calidad, se compone de elementos subjetivos y que están sujetos
al contexto del problema. Para medir la mantenibilidad de un producto de
software, se suelen utilizar métricas cuantitativas y cualitativas.
Esta característica suele responder a cómo el software es capaz de lidiar
con los siguientes problemas:
\begin{enumerate}
	\item qué tan fácil es la modificación del producto por parte de terceros
	\item qué tan fácil es analizar el producto de software
	\item qué tan fácil es extender el producto de software una vez terminado el desarrollo principal
	\item qué tan completa es la capacidad de pruebas del producto
	\item qué tan modular es el producto de software
	\item qué tanto reuso de código ocurre dentro del producto
    \item qué tanto código es posible reutilizar para extender el producto
\end{enumerate}

La mantenibilidad afecta generalmente a los desarrolladores,
puesto que el usuario final no es perjudicado directamente por la calidad
del código fuente mientras el software cumpla con los requerimientos de usuario.
Es frecuente escuchar sobre proyectos que han debido ser paralizados
debido a la baja calidad en su código, lo cual no permite continuar su desarrollo y mantención.
