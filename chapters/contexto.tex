%hablar sobre calidad, mantenibilidad y su principal función en el desarrollo.

\chapter{Contexto}

%sobre calidad
\section{Calidad de Software}
Calidad de Software es un tema tremendamente importante en el proceso de desarrollo.
Hoy en día sabemos que un producto de software que no cuenta con los estándares de
calidad adecuados, no goza de propiedades como mantenibilidad, seguridad, usabilidad, etc.
Si bien la calidad en general es un concepto subjetivo, se han desarrollado un gran
número de intentos por definir una base común con la cual se pueda evaluar un producto 
de software o que sirva para definir un desarrollo apropiado con el fín de terminar
con un producto confiable. Estos intentos generalmente apuntan a definir un modelo
de calidad que agrupe las principales características que un producto debe tener.

La ISO, IEC y IEEE definen calidad a través de 6 distintas alternativas~\cite{5276043}:
\begin{enumerate}
    \item El grado en el cual un sistema, componente o proceso cumple con los requerimientos especificados.
    \item La capidad de un producto, servicio, sistema o proceso para cumplir con las necesidades, expectativas
    o requerimientos del usuario o cliente.
    \item El conjunto de características de una entidad que le confieren su habilidad de satisfacer los requerimientos
    declarados y además los implícitos.
    \item Conformidad en las expectativas del usuario, conformidad en los requerimientos del usuario, satisfacción del cliente,
    confiabilidad y el nivel presente de defectos.
    \item El grado en el cual un conjunto inherente de características cumple con los requerimientos.
    \item El grado en el cual un sistema, componente o proceso cumple con las necesidades o expectativas de un ciente o usuario.
\end{enumerate}

Como se puede observar en las definiciones, calidad no tiene una definición universal e incluso se podría argumentar
que es un tema de carácter filosófico~\footnote{\url{http://en.wikipedia.org/wiki/Quality_(philosophy)}}.

\section{Mantenibilidad}

%sobre mantenibilidad
La mantenibilidad también es un tema recurrente en ingeniería de software. Al igual que calidad, se compone de elementos
subjetivos y que están sujetos al contexto del problema. Para medir la mantenibilidad de un producto de software, se suelen
utilizar métricas cuantitativas y cualitativas. Estas métricas usualmente indican que tan mantenible es el producto. 
Esto puede implicar diversos hechos, por ejemplo, que tan fácil es modificar el código fuente del software, que tan seguro
es la modificación del código fuente sin dañar el funcionamiento correcto del sistema, que tan fácil es entender el código
fuente del software con el fin de modificarlo o extenderlo, etc.
Esta característica de calidad afecta generalmente a los desarrolladores, puesto que el usuario final no es afectado directamente
por la buena calidad del código fuente, mientras el software cumpla con los requerimientos de usuario. Sin embargo es frecuente
escuchar sobre proyectos que han debido ser paralizados producto de falta de calidad en su código, lo cual no permite seguir
manteniéndolo.

