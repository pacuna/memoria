%hablar sobre calidad, mantenibilidad y su principal función en el desarrollo.

\chapter{Contexto}

%sobre calidad
\section{Calidad de Software}
Calidad de Software es un tema tremendamente importante en el proceso de desarrollo.
Hoy en día sabemos que un producto de software que no cuenta con los estándares de
calidad adecuados, no goza de propiedades como mantenibilidad, seguridad, usabilidad, etc.
Si bien la calidad en general es un concepto subjetivo, se han desarrollado un gran
número de intentos por definir una base común con la cual se pueda evaluar un producto 
de software o que sirva para definir un desarrollo apropiado con el fín de terminar
con un producto confiable. Estos intentos generalmente apuntan a definir un modelo
de calidad que agrupe las principales características que un producto debe tener.

La ISO, IEC y IEEE definen calidad a través de 6 distintas alternativas~\cite{5276043}:
\begin{enumerate}
    \item El grado en el cual un sistema, componente o proceso cumple con los requerimientos especificados.
    \item La capidad de un producto, servicio, sistema o proceso para cumplir con las necesidades, expectativas
    o requerimientos del usuario o cliente.
    \item El conjunto de características de una entidad que le confieren su habilidad de satisfacer los requerimientos
    declarados y además los implícitos.
    \item Conformidad en las expectativas del usuario, conformidad en los requerimientos del usuario, satisfacción del cliente,
    confiabilidad y el nivel presente de defectos.
    \item El grado en el cual un conjunto inherente de características cumple con los requerimientos.
    \item El grado en el cual un sistema, componente o proceso cumple con las necesidades o expectativas de un ciente o usuario.
\end{enumerate}

Como se puede observar en las definiciones, calidad no tiene una definición universal e incluso se podría argumentar
que es un tema de carácter filosófico~\footnote{\url{http://en.wikipedia.org/wiki/Quality_(philosophy)}}.

Un modelo clásico de calidad de producto, que puede ser aplicado en productos
de software es el de Garvin~\cite{Garvin:1984}. Garvin presenta los siguientes
enfoques que se pueden utilizar dentro de calidad:
\begin{itemize}
    \item Enfoque trascendente
    \item Enfoque basado en el producto
    \item Enfoque basado en el usuario
    \item Enfoque de fabricación
    \item Enfoque basado en valor
\end{itemize}

El enfoque trascendente es el más difuso. Se refiere a la propiedad inherente
e indefinible de un producto, con la cual cumple con una alta calidad. Podría
decirse que es casi una característica intuitiva con la cual se sabe si un
producto es de calidad.

El enfoque basado en producto describe diferencias en la cantidad de algunos
atributos deseados en el producto. Por lo tanto a diferencia del enfoque
trascendente, este enfoque puede ser medido. Asumimos que se conoce
y se puede describir lo que se desea.
Este enfoque es complicado en productos de software puesto que algunas métricas
pueden no existir o ser muy difíciles de medir. Un ejemplo es la mantenibilidad.
Si se asocia mantenibilidad al esfuerzo requerido para completar un cambio, este
esfuerzo es difícil de medir a través de diversos desarrolladores ya que no están
constantemente midiendo el tiempo en sus respectivas tareas. Además depende de 
otros factores como la complejidad del cambio.

En el enfoque basado en usuario, se asume que el producto que satisface
las necesidades del usuario de mejor manera, tiene la mejor calidad. El énfasis
no está en los requerimientos, sino en la impresión subjetiva de los usuarios.

Una visión más interna se observa en el enfoque de fabricación. En este enfoque
calidad se define como la conformidad con respecto a los requerimientos especificados.
Se debe asumir que siempre será posible definir un requerimiento así como la desviación
del producto real con respecto a este requerimiento. En este enfoque las métricas
concretas, por ejemplo defectos por linea de código son más útiles mientras puedan
relacionarse con algun requerimiento especificado.

Finalmente en el enfoque basado en valor se asignan costos a la conformidad y no
conformidad con respecto a requerimientos, se comparan con los beneficios del
producto y con estos datos se calcula su valor.

Garvin sugiere que estos distintos enfoques puede ser útiles durante distintas
etapas del ciclo de vida del producto. Por ejemplo, al inicio del ciclo de 
vida, durante la incepción del producto, debemos enfocarnos más en el usuario
y en el valor, con el fin de entender que es lo que tiene más valor para
los clientes y usuarios. O por ejemplo, cuando se está construyendo el producto,
debemos concentrarno más en la fabricación, con el fín de asegurar que se está
construyendo un producto adecuado con respecto a la especificación.

\subsection{Calidad de procesos}
Otra parte importante de calidad que es fuertemente estudiada es la calidad de 
procesos. La principal idea detrás de este enfoque, es que mientras de más
calidad sean los procesos, de más calidad serán los productos.

Un estándar altamente utilizado en la calidad de procesos es la ISO 9000. La idea
consiste en establecer un sistema de gestión de calidad en una compañía de
manera que la calidad resultante del producto también sea alta. No tiene que ver
directamente con la calidad del producto, sino con los requerimientos de calidad
de la compañía que produce el producto. 
La introducción de este tipo de estándares dentro de una compañía de software
puede beneficiarla al hacer que los procesos de aseguramiento de calidad sean
más explícitos y claros~\cite{Wagner:2013}.

Otras iniciativas similares, que se enfocan específicamente en mejorar los
procesos de desarrollo de software son CMMI y SPICE.
Estos estándares se basan en la premisa de que existe un proceso ideal, 
el cual es descrito en los estándares y que cada compañía debe alcanzar.
Existen nivéles de madurez y de capacidad partiendo desde procesos caóticos,
hasta procesos altamente estandarizados y optimizados.

La utilización de este enfoque para evaluar continuidad de desarrollo sufre
de ciertas carencias. Para empezar se parte con la premia de que mientras
mejores sean los procesos, mejor será la calidad del producto. Esto quizás
es claro en fabricación de productos, pero no tan claro en desarrollo de
software. Por ejemplo en~\cite{Jones:2000:SAB:335582} se presenta una 
investigación que buscaba la relación entre el nivel CMM (predeceso de CMMI) 
de una compañía, y el número de defectos por punto de función entregado en sus productos.
Como resultado se encontró que mientras el nivel aumentaba, los errores diminuían,
lo cual confirmaría la premisa. Sin embargo en los extremos, las mejores
compañías que estaban en nivel CMM 1 (el peor), producían software con menos
defectos que las peores companías con CMM 5 (el mejor). Por lo tanto
existe otros factores que estaban influyendo en la tasa de defectos.

Es importante analizar la calidad de procesos para lograr entregar un producto
de software de alta calidad, sin embargo como muchos otros factores influyen
en la industria del software, este trabajo está más enfocado en estudiar
la calidad del producto, y a través de esta calidad encontrar un modelo que
permita evaluar también la continuidad de desarrollo.

\section{Mantenibilidad}

%sobre mantenibilidad
La mantenibilidad también es un tema recurrente en ingeniería de software.
Al igual que calidad, se compone de elementos subjetivos y que están sujetos
al contexto del problema. Para medir la mantenibilidad de un producto de
software, se suelen utilizar métricas cuantitativas y cualitativas. Estas
métricas usualmente indican que tan mantenible es el producto.
Esto puede implicar diversos hechos, por ejemplo, que tan fácil es modificar
el código fuente del software, que tan seguro es la modificación del código
fuente sin dañar el funcionamiento correcto del sistema, que tan fácil es entender el código
fuente del software con el fin de modificarlo o extenderlo, etc.
Esta característica de calidad afecta generalmente a los desarrolladores,
puesto que el usuario final no es afectado directamente por la buena calidad
del código fuente, mientras el software cumpla con los requerimientos de usuario.
Sin embargo es frecuente escuchar sobre proyectos que han debido ser paralizados
producto de falta de calidad en su código, lo cual no permite seguir
manteniéndolo.

