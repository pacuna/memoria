%hablar sobre calidad, mantenibilidad y su principal función en el desarrollo.

\section{Contexto}

%sobre calidad

Calidad de Software es un tema tremendamente importante en el proceso de desarrollo.
Hoy en día sabemos que un producto de software que no cuenta con los estándares de
calidad adecuados, no goza de propiedades como mantenibilidad, seguridad, usabilidad, etc.
Si bien la calidad en general es un concepto subjetivo, se han desarrollado un gran
número de intentos por definir una base común con la cual se pueda evaluar un producto 
de software o que sirva para definir un desarrollo apropiado con el fín de terminar
con un producto confiable. Estos intentos generalmente apuntan a definir un modelo
de calidad que agrupe las principales características que un producto debe tener.

La ISO, IEC y IEEE definen calidad a través de 6 distintas alternativas~\cite{5276043}:
\begin{enumerate}
    \item El grado en el cual un sistema, componente o proceso cumple con los requerimientos especificados.
    \item La capidad de un producto, servicio, sistema o proceso para cumplir con las necesidades, expectativas
    o requerimientos del usuario o cliente.
    \item El conjunto de características de una entidad que le confieren su habilidad de satisfacer los requerimientos
    declarados y además los implícitos.
    \item Conformidad en las expectativas del usuario, conformidad en los requerimientos del usuario, satisfacción del cliente,
    confiabilidad y el nivel presente de defectos.
    \item El grado en el cual un conjunto inherente de características cumple con los requerimientos.
    \item El grado en el cual un sistema, componente o proceso cumple con las necesidades o expectativas de un ciente o usuario.
\end{enumerate}

Como se puede observar en las definiciones, calidad no tiene una definición universal e incluso se podría argumentar
que es un tema de carácter filosófico~\footnote{\url{http://en.wikipedia.org/wiki/Quality_(philosophy)}}.
%sobre mantenibilidad

