%e.a para continuidad de desarrollo
\chapter{Estado del arte}

\section{Estudios acerca de Mantenibilidad de Software}

Ya en la década de los años noventa, se consideraba que la mantenibilidad
de software era un tema fundamental y que podía generar un impacto notorio
en el costo de mantenimiento de un producto de software. En \cite{Oman:1992}
se definen ciertas métricas para realizar mediciones de mantenibilidad y se
presenta un índice de mantenibilidad que permite unificar estas métricas en
una sola.

Uno de los estudios conocidos de esta época se puede encontrar en \cite{Coleman:1996}.
En este trabajo altamente citado se presentan guías  para automatizar el análisis de
mantenibilidad con el fin de apoyar la toma de decisiones relacionadas con el
software en cuestión. Se evaluaron 5 métodos utilizando datos reales y luego
se menciona como estos métodos pueden ser utilizados en un ambiente industrial.
Los modelos aplicados fueron: modelos jerárquicos multidimensionales, modelos
de regresión polinomial, medidas de complejidad basadas en entropía, análisis
de componentes principales y análisis de factores.

Otro trabajo importante llevado a cabo en esta década es
presentado en \cite{West:1996}. Se presentan algunas definiciones
relativas a mantenibilidad de software así como la importancia que tiene
este atributo en cualquier sistema. Además se presenta un método básico
para evaluar mantenibilidad utilizando dos indicadores principales: la
medida de mantenibilidad (MM) y el índice de mantenibilidad (MI). Este último
es utilizado hasta hoy en día para tener una referencia rápida del comportamiento
estático de un sistema. 

Ya para los siguientes años, el tema de mantenibilidad de software es considerado
de alta importancia y se puede encontrar diversos estudios que resumen y presentan
el estado del arte del área. A continuación se nombran los más relevantes
para esta memoria.

En \cite{survey} se presentan los principales factores que afectan a la mantenibilidad de
software. Estos factores son extraídos de diferentes autores los cuales 
proponen diferentes modelos de mantenibilidad de software.
En el desarrollo del reporte se presentan factores tales como 
analizabilidad, modificabilidad, estabilidad, capacidad de pruebas, modularidad,
descriptividad, consistencia, simplicidad, etc. Estos son tomados de modelos
conocidos tales como el modelo de calidad de la ISO 9126, el modelo de calidad
de Boehm, modelo de McCall, el modelo Fuzzy, modelo MEMOOD, etc.

En \cite{roadmap} se describen guías para estudiar la mantenibilidad de software
utilizando métricas basadas en orientación a objetos. Para este estudio
se generó un catálogo de investigaciones importantes relevantes utilizando
ciertas heurísticas. Se encontraron 606 métricas de las cuales 570 eran métricas
orientadas a objectos y 71 métricas orientadas a aspectos. Otro resultado
interesante que surgió del estudio fueron los tópicos encontrados al buscar
investigaciones relacionadas con mantenibilidad. Estos fueron limitaciones
de arquitectura de software, herencia, cohesión, acoplamiento, complejidad
y tamaño. De estos conceptos se encontró que cohesión y acoplamiento eran
los tópicos mayormente investigados en la literatura con respecto a mantenibilidad.

Otro estudio relacionado con mantenibilidad enfocada en orientación a objectos
se puede encontrar en \cite{pastDecade}. En este trabajo se revisan estudios
de métricas de mantenibilidad de software orientado a objectos correspondientes
a la década pasada (2003-2012). Del estudio se concluyó que los métodos
basados en análisis de regresión (RA) fueron los mayormente empleados para
evaluación durante la década. Otros modelos de calidad existentes tales como
ISO 9126 o el modelo de McCall son escasamente utilizados en el desarrollo
de modelos de mantenibilidad. La mayoría de los estudios obtenidos en la investigación
hacen uso de las métricas orientadas a objetos existentes sin una revisión
y adaptación crítica antes de ser utilizadas para desarrollar modelos.

Otro estudio de métricas existentes para software orientado a objectos se puede
encontrar en \cite{TowardsACatalog}. En este trabajo se discuten un rango de
opciones para categorizar métricas. Lo que se buscar es solucionar el problema
de la falta de información útil acerca de métricas para mantenibilidad de 
software orientado a objetos que apoye la toma de decisiones acerca de cuales
de estas métricas debiesen ser adoptadas en estudios académicos o incluso
en actividades del día a día en la industria del software. Este trabajo
es una continuación de \cite{roadmap}, en el cual se encontraron 570
métricas para software OO. El objetivo es categorizar este conjunto de métricas
 para facilitar la implementación de un modelo para mantenibilidad. 
 Se generaron un total de 15 categorías que van desde métricas de evaluación
 hasta métricas de herramientas de ayuda.

 En \cite{SystematicReview} se presentan los resultados luego de realizar
 una revisión de la literatura para identificar métricas contemporáneas 
 asociadas con mantenibilidad de software y propuestas para tecnologías
 orientadas a aspectos y orientadas a características. En este trabajo
 se puede encontrar la lista de métricas y propiedades medibles para programación
 orientada a aspectos y orientada a características. Con estas métricas se 
 elaboró un catálogo unificado de métricas aplicables a ambas tecnologías
 y además se presenta sus principales referencias.

 Otro tipo de tecnología relevante es la orientada a servicios. En
 \cite{ResearchOnMaintainability} se analizan factores que afectan en la mantenibilidad
 de software y presenta un método de evaluación para mantenibilidad de software
 orientado a servicios. Se divide la mantenibilidad de software en analizabilidad,
cambiabilidad, estabilidad y capacidad de prueba como índice de evaluación. Para
cada uno de estos atributos de calidad se presentan métricas que pueden ayudar
a mejorar el diseño de software y seleccionar mejores estrategias de mantenimiento.

Otra tópico importante corresponde a los modelos de predicción de mantenibilidad.
Estos modelos son herramientas matemáticas que de acuerdo a ciertas métricas
estudiadas en el software, pueden entregar predicciones del comportamiento futuro
del software en lo que respecta a su mantenimiento.
En \cite{Riaz:2009} se presenta una revisión sistemática de predicción de mantenibilidad
de software y de métricas utilizando distintas pregunta para conducir su investigación.
Algunas de estas preguntas fueron: ¿Qué medidas han sido utilizadas para
medir la precisión de las predicciones de mantenibilidad de aplicaciones de software?,
¿Qué factores y métricas han sido investigados como predictores de mantenibilidad
para aplicaciones de software?. Este estudio está más focalizado en estudiar
predictores de mantenibilidad y al igual que la mayoría de este tipo de investigaciones,
se realizaron consultas adecuadas a un repositorio de investigaciones para
analizar los resultados que se repiten con mayor frecuencia y que pudiesen responder
a las preguntas planteadas.
Uno de los resultados de la revisión es que no existe un modelo de predicción
obvio para mantenibilidad. Se encontraron distintos tipos de modelos para predecir
mantenibilidad, la mayoría basados en algoritmos de regresión y de validación
cruzada. También se observó que los predictores más utilizados fueron aquellos
basados en tamaño, complejidad y acoplamiento.

Una aplicación más directa de modelo de calidad establecido se puede encontrar en
\cite{Baggen:2012}. En este trabajo se entrega una descripción del método utilizado
por el \textit{Software Improvement Group} para analizar calidad de código 
enfocándose en mantenibilidad. Este método utiliza un modelo de medición
basado en la conocida ISO/IEC 9126, específicamente en la definición de mantenibilidad
y métricas de código fuente. Utilizando las métricas adaptadas para este modelo
se genera un repositorio que luego es utilizado como \textit{benchmark} en
el cual se acumulan distintos resultados de evaluaciones los que cuales permiten
ir calibrando el modelo de medición.

Se han realizado numerosos estudios en el campo de sistemas OO. El tema
de mantenibilidad también ha sido crucial en estos estudios. En \cite{Kumar:2011}
se realiza una revisión de diversos modelos de mantenibilidad 
encontrados en la literatura. En este estudio sólo se han considerado
aquellos trabajos en los cuales es utiliza OO. Para el análisis se
tomaron investigaciones de diversas fuentes. En el trabajo se detallan
las principales técnicas encontradas en los artículos.
