\documentclass[letterpaper]{article}
\usepackage{natbib,alifeconf}
\usepackage[utf8]{inputenc}
\title{Modelo de calidad para evaluar continuidad de desarrollo}
\author{Pablo Acuña\\
\mbox{}\\
Universidad Técnica Federico Santa María \\
pacuna@alumnos.inf.utfsm.cl}

\begin{document}
\maketitle

\section{Introducción}

En este trabajo se propone un modelo de calidad para evaluar la continuidad
de desarrollo de un producto de software, así como las tareas necesarias
para llevar a cabo una evaluación de calidad.
La base de este modelo se encuentra en la serie ISO 25000
específicamente en la subdivisión de modelo de calidad.
Se han realizado ajustes para adaptar y refinar el modelo con el fin de evaluar la mantenibilidad de un producto de software.
Esta característica influye directamente en la continuidad de desarrollo y puede
ser descrita a través subcaracterísticas y métricas que sean capaces de cuantificarlas.

Se presenta también una aplicación del modelo generado, sobre un producto de software
real, con el fin de mostrar los pasos principales a la hora de llevar a cabo
una evaluación de calidad y para ejemplificar el uso del framework propuesto. Este se compone
del modelo de mantenibilidad y de las tareas principales tras el proceso de evaluación.

\section{Contexto}
% mini calidad
% mini mantenibilidad
La calidad de Software es uno de los temas más relevantes en la industria actual. Muchos de estos productos
son de alto impacto y su correcto funcionamiento es crítico en muchas actividades. Este funcionamiento
puede ser descrito a través de un gran número de características y subcaracterísticas, las cuales al ser agrupadas
de una manera sistemática, conforman los modelos de calidad. Una de estas características es la Mantenibilidad, y ligada
a esta se encuentra la continuidad de desarrollo. Estos elementos son indicadores de como el producto puede lidiar
con los siguientes problemas (entre otros):
\begin{enumerate}
	\item qué tan fácil es la modificación del producto por parte de terceros
	\item qué tan fácil es analizar el producto de software
	\item qué tan fácil es extender el producto de software una vez terminado el desarrollo principal
	\item qué tan completa es la capacidad de pruebas del producto
	\item qué tan modular es el producto de software
	\item qué tanto reuso de código ocurre dentro del producto
    \item qué tanto código es posible reutilizar para extender el producto
\end{enumerate}

Para evaluar y mejorar la continuidad de desarrollo de un producto a través de su mantenibilidad, se pueden optar por
varias alternativas, siendo las principales utilizar algún modelo de calidad o de madurez de procesos.
Los modelos de calidad han sido utilizados en la industria del software desde hace varias décadas, encontrándose diversos
enfoques en la literatura. Por otra parte el estudio de la madurez de procesos también puede ser utilizado para
evaluar mantenibilidad pero con un menor grado de confianza~\cite{Jones:2000:SAB:335582}.

\section{Problema}
El problema principal consiste en la falta de un modelo de calidad apropiado
que permita evaluar la continuidad de desarrollo de un producto de software.
Este modelo debe estar enfocado en los aspectos que más contribuyan a la mantenibilidad
del producto, ya que, como se mencionó previamente, esta es la característica que más
influye en la continuidad de desarrollo.
Cuando este desarrollo es tomado por terceros, la mantenibilidad
del software se torna crítica, ya que afecta directamente a todos los procesos y tareas
que este desarrollo conlleva, ya sea la mantención, eliminación de fallas, análisis de código,
extensión del software, reutilización de módulos o funciones, etc.

Además de obtener un conjunto de métricas para evaluar las subcaracterísticas
correspondientes, se debe diseñar un proceso de evaluación que sirva como guía a los
evaluadores y que permita realizar las tareas de evaluación de manera sistemática y organizada, 
siempre tomando en cuenta el contexto del problema.

\section{Estado del Arte}
El modelo de evaluación para continuidad de desarrollo
 de este trabajo en particular está basado en un modelo de calidad. Otra opción hubiese sido presentar un modelo
de madurez de procesos basado por ejemplo en CMMI o SPICE. La elección se debe
principalmente a la cantidad de referencias que podemos encontrar en la literatura
con respecto a prácticas de evaluación y cómo crear buenos planes de calidad que mejoren la mantenibilidad
del producto de software. Al estudiar esta literatura se encontró que la gran mayoría
de trabajos exponen visiones e ideas que convergen en modelos de calidad típicos, más
que en modelos de mejoramiento de procesos.
Dicho esto, el estudio de modelos de madurez puede ser un excelente complemento a los modelos de
calidad basado en métricas de software y cualquier empresa que desee llevar sus productos
y procesos al siguiente nivel, debería estudiarlos.

A través de los años, se han ido presentando diversas propuestas para aplicar modelos
de calidad con el fín de evaluar la mantenibilidad de software. Algunos trabajos
notables son: \cite{Coleman:1994} y \cite{Oman:1992}. Cabe mencionar que no todos estos trabajos
están basados puramente en asignar valores a rangos de métricas, si no que también
contienen modelos matemáticos complejos tales como modelos de regresión polinomial,
medidas de complejidad basadas en entropía, análisis de componentes principales, 
análisis de factores, etc.
En algunos trabajos se presentan métodos que aún prevalecen y son importantes en
la industria del software actual, por ejemplo el índice de mantenibilidad (MI) 
presentado en \cite{West:1996}.
Otros autores intentan definir las principales características que influyen en la
mantenibilidad de un producto de software. De trabajos como \cite{survey} se obtienen
características que forman parte de modelos actuales tales como la modificabilidad,
estabilidad, capacidad de pruebas, modularidad, etc. Estas características son 
tomadas luego por modelos como el presentado en la ISO 9126, modelo de Boehm,
modelo de McCall, etc.
Estudios relacionados con orientación a objetos son \cite{roadmap}, \cite{pastDecade} y \cite{TowardsACatalog}. Estos trabajos
están enfocados en evaluar software orientado a objetos utilizando diversos
métodos.
Todos estos distintos trabajos han servido para ir refinando cada vez los modelos
utilizados actualmente, dandole mayor importancia a la características que 
afectan de manera más directa y que son mas universales a través de los distintos
productos y contextos en la industria.

\section{Propuesta}

La propuesta presentada en este trabajo consiste en la generación de un framework
para evaluar la continuidad de desarrollo de un producto de software. Este
framework tiene su base en la serie ISO 25000. De esta serie se han extraído un 
subconjunto de características y subcaracterísticas
que permiten construir un modelo que apunta a la evaluación de la continuidad de desarrollo.
Esta ISO también ha sido utilizada como base para generar un esquema de trabajo y de
las tareas fundamentales a la hora de llevar a cabo una evaluación sobre un producto real.

\subsection{ISO 25000}
Esta división de la ISO es también conocida como SQuaRE (Software product
quality Requirements and Evaluation) y se compone de 5 subdivisiones:
\begin{itemize}
    \item ISO/IEC 2500n - División de Gestión de Calidad
    \item ISO/IEC 2501n - División de Modelo de Calidad
    \item ISO/IEC 2502n - División de Medición de Calidad
    \item ISO/IEC 2503n - División de Requerimientos de Calidad
    \item ISO/IEC 2504n - División de Evaluación de Calidad
\end{itemize}

Cada una de estas subdiviones entrega guías para tareas específicas dentro
de los procesos de evaluación de software. 
Para la construcción del framework de evaluación se utilizaron 
principalmente la división de Modelo de Calidad, de la cual
se obtuvo un conjunto de características  y subcaracterísticas enfocadas en la mantenibilidad del 
producto, y la división de Evaluación de Calidad, de la cual se obtuvieron
guías para aplicar el modelo obtenido y hacer una evaluación de calidad
sobre un producto de software real.

El modelo ISO/IEC 25010 presenta a la mantenibilidad como una de sus 8 características
principales, por lo que facilita la elección de las subcaracterísticas al ser
entregadas junto al modelo. 
Esta característica se compone de las siguientes subcaracterísticas:

\begin{itemize}
	\item Modularidad
	\item Reusabilidad
	\item Analizabilidad
	\item Modificabilidad
	\item Capacidad de pruebas
\end{itemize}

Con estas subcaracterísticas en mente, se puede comenzar a realizar la elección
de métricas que las describirán y que permitirán asignar valores reales a un 
producto de software. 

La implementación de modelo de mantenibilidad en esta propuesta consiste en
las siguientes subcaracterísticas y métricas:

\begin{itemize}
\item Modularidad
	\begin{itemize}
		\item Cohesión Relacional
		\item LCOM (Falta de cohesión en métodos)
		\item LCOM HS(Falta de cohesión en métodos Henderson-Sellers)
		\item Acoplamiento Eferente
		\item Acomplamient Aferente
	\end{itemize}
\item Reusabilidad
	\begin{itemize}
		\item Cantidad de Código Duplicado
	\end{itemize}
\item Modificabilidad
	\begin{itemize}
		\item Instabilidad
		\item Complejidad Ciclomática 
		\item Índice de Mantenibilidad
        \item Produndidad (lado cliente)
        \item Número de parámetros por función (lado cliente)
	\end{itemize}
\item Capacidad de pruebas
	\begin{itemize}
		\item Cubrimiento de pruebas de funciones críticas y no críticas
	\end{itemize}
\end{itemize}

La segunda parte de la propuesta consiste en las tareas principales
que se deben llevar a cabo el proceso de evaluación. Estas consisten en:

\begin{enumerate}
    \item Establecer los requerimientos de la evaluación
        \begin{enumerate}
                \item Establecer el propósito de la evaluación
                \item Obtener los requerimientos de calidad del producto de software (si es que existen)
                \item Identificar las partes del producto que serán sometidas a la evaluación
                \item Definir el rigor de la evaluación (tomar en cuenta presupuestos, tiempos, propósito, etc)
        \end{enumerate}
    \item Diseñar la evaluación
        \begin{enumerate}
                \item Generar el plan de actividades de evaluación
                    Se debe tener en cuenta el presupuesto, métodos de evaluación, herramientas de evaluación, 
                    calendarización de actividades, recursos disponibles, etc.
        \end{enumerate}
    \item Ejecutar la evaluación
        \begin{enumerate}
                \item Realizar las mediciones utilizando el modelo entregado
                \item Aplicar criterios de decisión para las medidas utilizando el modelo entregado
                \item Aplicar criterios de decisión para la evaluación (tomar en cuenta los requerimientos de calidad
                    del producto)
        \end{enumerate}
    \item Concluir la evaluación
        \begin{enumerate}
            \item Revisar los resultados de la evaluación
            \item Entregar los datos de evaluación
        \end{enumerate}
\end{enumerate}

Estas actividades fueron seleccionadas utilizando las recomendaciones entregadas en la ISO 2504n~\cite{25040}.
Con estas tareas, y junto al modelo definido, se tiene un framework con el cual podemos llevar a cabo
una evaluación de calidad específicamente para evaluar la continuidad de desarrollo de un producto de software
usando como puente las métricas que definen su mantenibilidad.

\section{Validación}

En esta sección se presenta la aplicación del framework para evaluar mantenibilidad
sobre un producto de software real. En esta versión resumida sólo se presentan
los resultados de la aplicación del modelo de calidad. La versión completa así como
la descripción de las tareas de evaluación, se pueden encontrar en el informe completo.

\subsection{Resultados de la primera evaluación}

\subsubsection{Cohesión Relacional}

Los resultados para esta métrica se pueden observar en el cuadro número~\ref{table:CR}.

\begin{table}
    \centering
\tiny
    \begin{tabular}{|l|l|l|}
    \hline
       \bf{Assemblies}       & \bf{Cohesión Relacional} & \bf{Nombre}           \\ \hline
       SaeFramework2013 & 1.4                   & SaeFramework2013 \\ \hline
    \end{tabular}
    \caption{Cohesión relacional}
    \label{table:CR}
\end{table}

\subsubsection{LCOM}

Los resultados de esta métrica se pueden ver en el cuadro~\ref{table:LOCM}.

\begin{table}
    \centering
\tiny
    \begin{tabular}{|l|l|}
    \hline
    \bf{Nombre}                                                         & \bf{LCOM}    \\ \hline
       Mosaq.SaeFramework.v2013.Vistas.ListadoAuditoriaTicket      &    0.97 \\ \hline
       Mosaq.SaeFramework.v2013.Negocio.CreacionOSTicket           &    0.82 \\ \hline
       Mosaq.SaeFramework.v2013.Negocio.EdicionTicket              &    0.78 \\ \hline
       Mosaq.SaeFramework.v2013.Datos.Conector                     &    0.68 \\ \hline
       Mosaq.SaeFramework.v2013.Negocio.ManutencionCentroCosto     &    0.5  \\ \hline
       Mosaq.SaeFramework.v2013.Negocio.ManutencionCoberturaSitios &    0.42 \\ \hline
       Mosaq.SaeFramework.v2013.Negocio.ManutencionZonas           &    0.42 \\ \hline
    \end{tabular}
	\caption{Principales tipos y su LCOM}
    \label{table:LOCM}
\end{table}

\subsubsection{Acoplamiento eferente}
Las mediciones para los $10$ tipos con mayor valor de acoplamiento eferente se pueden observar en el cuadro~\ref{table:EC}.
\begin{table}
    \centering
\tiny
    \begin{tabular}{|l|l|}
    \hline
    \bf{Nombre}                                                    & \bf{Tipos que utiliza} \\ \hline
       Mosaq.SaeFramework.v2013.Vistas.ConsultaVistas              &    89                  \\ \hline
       Mosaq.SaeFramework.v2013.Negocio.TicketAcciones             &    34                  \\ \hline
       Mosaq.SaeFramework.v2013.Negocio.ReglasNegocio              &    28                  \\ \hline
       Mosaq.SaeFramework.v2013.Negocio.CreacionOSTicket           &    22                  \\ \hline
       Mosaq.SaeFramework.Utilidades.TypeMap                       &    26                  \\ \hline
       Mosaq.SaeFramework.v2013.Datos.Conector                     &    24                  \\ \hline
       Mosaq.SaeFramework.v2013.Utilidades.Utilidades              &    21                  \\ \hline
       Mosaq.SaeFramework.v2013.Standard.CmdbComponentes           &    20                  \\ \hline
       Mosaq.SaeFramework.v2013.Standard.GruposResolutoresEmpresas &    19                  \\ \hline
       Mosaq.SaeFramework.v2013.Standard.ZonasEmpresas             &    19                  \\ \hline
    \end{tabular}
    \caption{Top 10 de tipos y su acoplamiento eferente}
    \label{table:EC}
\end{table}

\subsubsection{Instabilidad}

Para el cálculo de esta medida se utilizó el promedio del Top 10 de tipos con mayor acoplamiento eferente y el promedio del Top 10 de los tipos con mayor acoplamiento aferente. Estos valores se pueden obtener utilizando los datos de las tablas anteriores. A continuación se presenta el resultado.

$$I = 30.2 / (30.2 + 102.3)$$
$$I = 0.23$$
\subsubsection{Complejidad ciclomática}

En el cuadro~\ref{table:CC} se pueden observar los 10 métodos con mayor complejidad ciclomática.

\begin{table}
    \centering
\tiny
    \begin{tabular}{|l|l|}
    \hline
    \bf{Nombre}                                                                          & \bf{Complejidad ciclomática} \\ \hline
       Vistas.ConsultaVistas.Lista                              &    77                          \\ \hline
       Negocio.TicketAcciones.Transferir                        &    10                          \\ \hline
       Negocio.TicketAcciones.ReabrirTicket                     &    8                           \\ \hline
       Negocio.TicketAcciones.Recatalogar                       &    5                           \\ \hline
       Negocio.TicketAcciones.ProgramarFechaAtencion            &    4                           \\ \hline
       Negocio.CreacionOSTicket   .grabarYcalcularFechasTickets &    4                           \\ \hline
       Utilidades.Validadores.digitoVerificador                       &    5                           \\ \hline
       Utilidades.StringEnum.Parse                              &    5                           \\ \hline
       Utilidades.Utilidades.IEnumerableToDataTable        &    5                           \\ \hline
       Datos.Conector.ejecutarProcedimientoCadenaAnidada     &    4                           \\ \hline
    \end{tabular}
    \caption{Top 10 de métodos y su complejidad ciclomática}
    \label{table:CC}
\end{table}


\subsubsection{Índice de mantenibilidad} 
En el cuadro~\ref{table:MI} se muestra el valor del índice de mantenibilidad para el código de software.
\begin{table}
    \centering
\tiny
    \begin{tabular}{|l|l|}
    \hline
    \bf{Nombre}      & \bf{Índice de mantenibilidad} \\ \hline
    SaeFramework2013 & 93                              \\ \hline
    \end{tabular}
    \caption{Índice de mantenibilidad}
    \label{table:MI}
\end{table}

\subsubsection{Complejidad Ciclomática lado cliente}

En el cuadro~\ref{table:JSCC} se pueden observar los archivos y directorios 
seleccionados para la evaluación, así como su respectiva complejidad por método.
\begin{table}
    \centering
\tiny
    \begin{tabular}{|l|l|}
    \hline
    \textbf{Archivo/Directorio}            & \textbf{Complejidad por método} \\ \hline
    app.js                        & 4.1                    \\ \hline
    BasePrincipal.js              & 3.0                    \\ \hline
    app/controllers/debug         & 1.2                    \\ \hline
    app/controllers/admin         & 1.7                    \\ \hline
    app/controllers/laboratorio   & 1.0                    \\ \hline
    app/controllers/main          & 1.9                    \\ \hline
    app/controllers/ordenservicio & 1.7                    \\ \hline
    app/controllers/tickets       & 2.2                    \\ \hline
    app/connection/wcf.js         & 4.1                    \\ \hline
    app/services/services.js      & 1.8                    \\ \hline
    \end{tabular}
    \caption{Elementos elegidos y su complejidad por método}
    \label{table:JSCC}
\end{table}

\subsubsection{Profundidad (lado cliente)}
En el cuadro~\ref{table:depth} se pueden observar los archivos y directorios 
seleccionados para la evaluación y su análisis respectivo de profundidad. Se 
eligió utilizar un valor de 5 como mínimo para que el 
elemento comience a ser crítico.
\begin{table}
    \centering
\tiny
    \begin{tabular}{|l|l|}
    \hline
    \textbf{Archivo/Directorio}            & \textbf{Profundidad}                 \\ \hline
    app.js                        & Sin profundidad mayor que 5 \\ \hline
    BasePrincipal.js              & Sin profundidad mayor que 5 \\ \hline
    app/controllers/debug         & Sin profundidad mayor que 5 \\ \hline
    app/controllers/admin         & Sin profundidad mayor que 5 \\ \hline
    app/controllers/laboratorio   & Sin profundidad mayor que 5 \\ \hline
    app/controllers/main          & Sin profundidad mayor que 5 \\ \hline
    app/controllers/ordenservicio & Sin profundidad mayor que 5 \\ \hline
    app/controllers/tickets       & Sin profundidad mayor que 5 \\ \hline
    app/connection/wcf.js         & Sin profundidad mayor que 5 \\ \hline
    app/services/services.js      & Sin profundidad mayor que 5 \\ \hline
    \end{tabular}
    \caption{Elementos elegidos y el match para profundidad seleccionado}
    \label{table:depth}
\end{table}
\subsubsection{Número de parámetros (lado cliente)}

En el cuadro~\ref{table:PN} se pueden observar los archivos y directorios 
seleccionados para la evaluación y su análisis respectivo del número de 
parámetros en sus funciones. Para esta métrica se decidió utilizar un mínimo 
de 5 parámetros para considerar un elemento como peligroso.
\begin{table}
    \centering
\tiny
    \begin{tabular}{|l|l|}
    \hline
    Archivo/Directorio            & Número de parámetros \\ \hline
    app.js                        & No sobrepasa             \\ \hline
    BasePrincipal.js              & No sobrepasa             \\ \hline
    app/controllers/debug         & No sobrepasa             \\ \hline
    app/controllers/admin         & No sobrepasa             \\ \hline
    app/controllers/laboratorio   & No sobrepasa             \\ \hline
    app/controllers/main          & No sobrepasa             \\ \hline
    app/controllers/ordenservicio & No sobrepasa             \\ \hline
    app/controllers/tickets       & No sobrepasa             \\ \hline
    app/connection/wcf.js         & No sobrepasa             \\ \hline
    app/services/services.js      & No sobrepasa             \\ \hline
    \end{tabular}
    \caption{Elementos elegidos y el match para número de parámetros seleccionado}
    \label{table:PN}
\end{table}

\subsection{Retroalimentación luego de la evaluación (lado servidor)}

El primer valor analizado es el de Cohesión relacional. El valor para este análisis es de 1.4. Este resultado se
encuentra dentro del rango considerado como correcto para una aplicación. Valores
más altos podrían indicar sobre-acoplamiento ya que la cohesión relacional nos entrega un promedio de relaciones internas por tipo dentro de un
paquete. Un paquete debe tener sus clases fuertemente relacionadas y estas no deberían estar relacionadas de manera
considerable con clases externas. 

La siguientes métrica corresponden a LCOM. Para esta medida existen rangos definidos que son considerados como correctos para un conjunto de tipos.

Se deben analizar los tipos que tengan un valor mayor a 0.8. En este caso existen 
2 tipos que sobrepasan ese valor. Estos tipos son:
\begin{itemize}
\item CreacionOSTicket (Clase)
\item ListadoAuditoriaTicket (Clase)	
\end{itemize}

Se han analizado estas clases y se concluyó que no es necesario realizar 
cambios sobre ellas ya que sólo están compuestas por \textit{getters} 
y \textit{setters}. Las clases de este estilo no son correctamente evaluadas 
por esta métrica ya que no tienen un comportamiento definido por relaciones 
entre sus métodos.

El valor de \textbf{acoplamiento eferente} podría revelar tipos que tienen 
muchas responsabilidades. Mientras mayor sea este valor, más entrelazado está 
el tipo con otras implementaciones. Si bien los valores obtenidos no son altos, 
se recomienda analizar los primeros tipos y verificar si pueden ser más modularizados.


El valor de \textbf{instabilidad} es de 0.23. Este valor puede variar entre 
0 y 1, donde 0 indica un paquete completamente estable y 1 un paquete 
completamente inestable. El valor obtenido en este caso es bastante aceptable 
e indica que el software se encuentra en un estado estable y tiene una buena 
modificabilidad. Cabe mencionar que este valor se obtiene analizando las 
métricas de acoplamiento descritas anteriormente, de esta manera este valor 
esta relacionado con el nivel de acoplamiento en el sistema.

Para estudiar la \textbf{complejidad ciclomática} se muestra una tabla con 
el top 10 de los métodos con la mayor complejidad. El estándar indica
que un método con una complejidad ciclomática mayor a 30 puede 
ser demasiado complejo y se debe estudiar la posibilidad de dividirlo en 
métodos más pequeños a menos que corresponda a código generado.
En este caso el análisis arrojó sólo un método con un valor superior a 30, 
el cual es \textbf{Listar()} dentro de \textbf{Mosaq.SaeFramework.v2013.Vistas.ConsultaVistas}.
Se analizó este método y se puede observar de que contiene un \textit{switch} 
con un gran número de \textit{cases}. Este \textit{switch} es el que genera 
un aumento en la complejidad ciclomática. Por lo tanto se recomienda estudiar 
este método y buscar alguna manera de refactorizar para obtener un código 
más mantenible. Algunas opciones para refactorizar un \textit{switch} 
se basan en utilizar diccionarios, mapas o simplemente separar las enumeraciones 
en sus propias clases.

El valor obtenido para el \textbf{índice de mantenibilidad} es de 93. Este 
valor puede estar entre 0 y 100 y mientras más alto indica una mejor 
mantenibilidad. En este caso el valor sugiere que el software es altamente 
mantenible bajo los estándares de esta métrica.

\subsection{Segunda evaluación}
Durante la segunda evaluación, se volvió a analizar el código fuente de ambos
módulos principales para determinar que las brechas encontradas fueran cerradas
correctamente. Se encontró que el software había sido mejorado y cumplía con
los estándares necesarios para ser certificado por el grupo evaluador.

Cabe mencionar que este trabajo sólo presenta una parte de la evaluación así
como algunos de los resultados. Para mayor información acerca de la aplicación completa
así como retroalimentación para el lado cliente, ver el informe de evaluación.

\section{Conclusiones}


%\begin{figure}[!htb]
%\begin{center}
%\includegraphics[width=2in]{lion.png}
%\caption{Sample picture caption.}
%\label{fig1}
%\end{center}
%\end{figure}


%\begin{table}[!hbt]
%\center{
%\begin{tabular}{|c|c|c|}\hline
%One & Two & Three\\ \hline\hline
%Yes & 0 & 1 \\
%Not  & 1 & 0 \\
%Maybe & 0.5 & 0.5 \\ \hline
%\end{tabular}
%}
%\vskip 0.25cm
%\caption{Sample table caption.}
%\end{table}


\footnotesize
\bibliographystyle{abbrv}
\bibliography{sample}


\end{document}
